\sectionmark{Wachstum der Bakterien I}

\section{Wachstum der Bakterien I}
\begin{enumerate}
	\item Nennen Sie fünf Makroelemente, die alle Organismen zum Leben brauchen und deren Quellen für Bakterien (dabei sollte mindestens 1 Metall sein)!
	\begin{table}
	\begin{center}
	\begin{tabular}{l l} 
	\toprule
		Element	&	Bereitstellung	in der Natur \\
		\midrule
		C			&	\ce{CO2}, organische Stoffe \\
		H			&	\ce{H2O}, organische Stoffe \\
		O			&	\ce{H2O}, \ce{O2}, organische Stoffe \\
		N			&	\ce{NH3}, \ce{NO3-}, \ce{N2}, organische Stoffe \\
		P			&	Phosphat \\
		S			&	\ce{H2S}, organische Stoffe, Sulfide \\
		\midrule
		K			&	Kaliumsalze \\
		Mg			&	Magnesiumsalze \\
		Na			&	\ce{NaCl}, Natriumsalze \\
		Ca			&	Salze \\
		\midrule
		Fe			&	\ce{FeS}, Eisensalze \\
	\bottomrule
	\end{tabular}
	\caption{Mikroelementen und ihre Quellen für Mikroorganismen}
	\label{tab:makroelemente}
	\end{center}
	\end{table}

	\item Nennen Sie einige Mikroelemente und erklären Sie, wofür diese Elemente benötigt werden!
	\begin{table} 
	\begin{center}
	\begin{tabular}{l l} 
	\toprule
		Element	&	Funktion in der Zelle \\
		\midrule
		Co			&	\ce{B12}, Transcaboxylase \\
		Cu			&	Atmung, CytC-Oxidase, Photosynthese\\
		Mn			&	Photosystem II, Superoxiddismutase \\
		Mo			&	Nitrogenase, Nitratreduktase, Formiat-DHG \\
		Ni			&	Hydorgenase, Co-F430 (Methanogene), \ce{CO}-Dehydrogenase \\
		Se			&	Formiat-DHG, Hydrogenase\\
		W			&	Formiat-DHG \\
		V			&	Vanadium-Nitrogenase \\
		Zn			&	Alkohol-DHG, RNA-/DNA-Polymerase \\
		Fe			&	Cytochrome, Katalasen, FeS-Proteine, alle Nitrogenasen \\
	\bottomrule
	\end{tabular}
	\label{tab:mikroelemente}
	\caption{Mikroelementen und ihre Quellen für Mikroorganismen}
	\end{center}
	\end{table}

	\item Stellen Sie sich vor, sie kultivieren \emph{E. coli}! Sie starten eine Kultur mit 3000 Zellen und kultivieren sie dann unter optimalen Bedingungen für 12 Stunden. Zeichnen Sie schematisch die Wachstumskurve dieser Kultur (bitte denken Sie an die Achsenbeschriftung und die Skalierung)!
	\item Die einfachste Methode zur Messung des Wachstums ist die Trübungsmessung. Würden Sie diese Methode anwenden, wenn Sie mit Streptomyceten arbeiten? Begründen Sie Ihre Antwort!
	\item Die Proteine FtsZ und MreB sind für die Zellteilung sehr wichtig. Wozu würde die Zerstörung dieser beiden Gene in \emph{E. coli} führen?
	\item Bei steigender Temperatur steigt die Wachstumsrate von Bakterien bis zum Optimum langsam an, dann fällt Sie schnell ab. Begründen Sie diesen Kurvenverlauf!
	\item Warum die Anpassung an niedrige Temperaturen bei Lipiden besonders wichtig? Nennen Sie einige Möglichkeiten dafür!
\end{enumerate}
