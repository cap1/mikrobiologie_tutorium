\sectionmark{Infektionsbiologie}

\section{Infektionsbiologie}
\begin{enumerate}
	\item Nennen Sie mindestens fünf bakterielle Krankheitserreger, die von ihnen verursachten Krankheiten und die systematischen Gruppen, zu denen diese Bakterien gehören!

	\begin{enumerate}[label=\arabic*)]
		\item \emph{Yersinia pestis} \hfill \\
			Pest, Gammaproteobacteria
		\item \emph{Vibrio cholerae} \hfill \\
			Cholera, Proteobacteria
		\item \emph{Bordetella pertussis} \hfill \\
			Keuchhusten, Proteobacteria
		\item \emph{Mycoplasma pneumoniae} \hfill \\
			Lungenentzündung, Firmicutes
		\item \emph{Bacillus anthracis} \hfill \\
			Milzbrand, Clostridien
	\end{enumerate}

	(Frei nach Frage \ref{item:boesebakterien}, Seite \pageref{item:boesebakterien})

	\item Nennen Sie einige typische Mikroorganismen, die den menschlichen Organismus normalerweise besiedeln und die dazugehörigen Organe und typischen Lebensbedingungen!

		\begin{table}[h!]
		\begin{center}
		\begin{tabular}{l l l} 
		\toprule
		Name 				&	Organ				& 	Lebensbedingungen	\\
		\midrule
		\emph{Helicobacter pylori}		&	Magen			&	sehr sauer		\\
		\emph{Enerococcus faecalis	}	&	Dickdarm		&	anaerob		\\
		\emph{Streptococcus sanguis}	&	Mundhöhle	&	gut		\\
		\emph{Streptococcus mutans}	&	Mundhöhle	&	säure tolreant durch \ce{H+}-Austoß		\\
		\emph{Propionibacterium acnes}&	Haut			&	periodisches Austrocknen	\\
		\emph{Staphylococcus aureus}	&	obere Atemwege	&	(windig)		\\
		\bottomrule
		\end{tabular}
		\caption{Typische Mikroorganismen die den Menschen besiedeln.}
		\label{tab:meinemitbewohneraufmir}
		\end{center}
		\end{table}

		In Tabelle \ref{tab:meinemitbewohneraufmir} befindet sich eine Aufstellung von,
		Mikroorganismen die den Menschen bewohnen.

	\item Warum ist Saccharose hochgradig kariogen, Fructose aber nicht? Welche Bakterien spielen bei der Karies eine wichtige Rolle?

		Saccharose kann von \emph{Streptococcus mutans} dazu verwendet werden,
		einen Biofilm aufzubauen in dem es sich noch besser einnisten kann.
		Dies schafft weiteren Platz für kariogenen Bakterien.

	\item Was sind die wichtigsten Schritte bei der Etablierung einer Infektion? Nennen Sie einige Faktoren, die den Bakterien dabei helfen!

		\begin{enumerate}[label=\arabic*)]
			\item Exponierung gegenüber Pathogenen
			\label{item:exposure}
			\item Anheftung der Pathogene an Haut, Schleimhaut
			\item Invasion durch das Epithel
			\item Koloniseriung und Wachstum, Erzeugung von Virulenzfaktoren
				\textrightarrow Schritt \ref{item:exposure}.\\
				\textrightarrow Toxische Effekte, Zell- und Gewebschäden, stärkere Exponierung \textrightarrow Schrit  \ref{item:exposure}.\\
		\end{enumerate}

		Grundsätzlich erleichtern bereits vorhandene Verletzungen das Eindringen von Pathogenen.
		Einige Pathogene können durch Toxin Zellen schwächen und sich so zugang verschaffen.
		Fimbrien, Pili und Schleimschichten erleichtern die Adhärenz.
		Bestimmte Pathogene besitzten eine spezifische Protein-Protein-Wechselwirkung mit dem Wirt,
		die die Adhärenz,
		als auch das Eindringen erleichtern.
\end{enumerate}
