\sectionmark{Aerober Stoffwechsel}

\section{Aerober Stoffwechsel}
\begin{enumerate}
	\item Warum ist es vorteilhaft, dass die Intermediate der Glykolyse phosphoryliert vorliegen?
		Phosphorylierte Intermediate koennen nicht durch die Membran transportiert werden und verbleiben somit in der Zelle
	\item Wie wird die Phosphorylierung des Glycerinaldehyd-3-Phosphats möglich?
		Reaktion in zwei Schritte aufgeteilt:
		Oxidation des Aldehyds zur Karbons\"aure durch NAD+, Vereinigung der Karbons\"aure mit dem Orthophosphat (\"uber Thioester-Zwischenprodukt). Durch die Kopplung wird der zweite, energetisch ung\"unstige, m\"oglich.
	\item Stellen Sie sich vor, sie sollen in \emph{E. coli} das Gen für die Phosphoglyceratmutase ausschalten. Wäre solch eine Mutante lebensfähig? Begründen Sie Ihre Antwort! 
		Die Zellen w\"aren Lebensf\"ahig, da Pyruvat zwar nicht mehr ueber die Glycolye, wohl aber noch \"uber den Entner-Duderhoff-Weg gewonnen werden kann 		    (Wenn auch mit schlechterer Ausbeute)
	\item Was sind die Spezifika des Pentosephosphatweges und des KDPG-Weges?
		\begin{itemize}
			\item Pentosephostphatweg: Erzeugt NAPDH und Ribose-5-Phosphat, Unterteilt in Oxidativer Teil (Phosphorylierung von Glucose, erzeugen von 6-Phosphoglucono-$\delta$-Lacton
			\item KDPG-Weg: Glucose $\Rightarrow$  2NADH + 1 ATP + 2 Pyruvat \"uber KDPG (2-Keto-3-desoxy-6phosphogluconat)
		\end{itemize}	
	
	\item Schreiben Sie die Reaktionen der Glykolyse auf, berücksichtigen Sie dabei ATP-Verbrauch und ATP-Bildung sowie die Gewinnung von \ce{NADH2} / \ce{FADH2}
		\begin{itemize}
			\item Glucose + ATP $\rightarrow$ Glucose-6-Phosphat + ADP (Hexokinase)
			\item Glucose-6-Phosphat $\rightarrow$ Fruktose-6-Phosphat (Glucose-6-Phosphat-Isomerase)
			\item Fruktose-6-Phosphat + ATP $\rightarrow$ Fruktose-1-6-Phosphat (Phosphofruktokinase) + ADP
			\item Fruktose-1-6-Phosphat $\rightarrow$ GA-3-P + DHAP (Aldolase)
			\item DHAP $\Longleftrightarrow$ GA-3-P Triosephosphatisomerase
			\item GA-3-P + $NAD^+$ $\rightarrow$ 1,3-Bisphosphoglycerat (Glycerinaldehyd-3-P-Dehydrogenase)
			\item 1,3-Bisphosphoglycerat + ADP $\rightarrow$ ATP + 3-Phosphoglycerat (Phosphoglyceratkinase)
			\item 3-Phosphoglycerat $\rightarrow$ 2-Phosphoglycerat (Phophoglyceratmutase)
			\item 2-Phosphoglycerat $\rightarrow$ PEP + $H_20$ (Enolase)
			\item PEP + ADP $\rightarrow$ Pyruvat + ATP (Pyruvatkinase)
		\end{itemize}
	\item Schreiben Sie die Reaktionen des Zitratzyklus auf, berücksichtigen Sie dabei ATP-Verbrauch und ATP-Bildung sowie die Gewinnung von \ce{NADH2} / \ce{FADH2} und die Freisetzung von \ce{CO2}!
		\begin{itemize}
			\item Acetyl-Coa + $H_2O$ + Oxalacetat $\rightarrow$ Citrat + CoA-SH (Citrat Synthase)
			\item Citrat $\rightarrow$ Isocitrat (Aconitase)
			\item Isocitrat + $NAD^+$ $\rightarrow$ $\alpha$-Ketoglutarat + $CO_2$ + NADH (Isocitrat-Dehydrogenase)
			\item $\alpha$-Ketoglutarat + $NAD^+$ + CoA-SH $\rightarrow$ Succinyl-Coa + NADH + $CO_2$ ($\alpha$-Ketoglutarat Dehydrogenase-Komplex)
			\item Succinyl-Coa + GDP $\rightarrow$ GTP + Succinat (Succinyl-CoA Synthethase)
			\item Succinat + FAD $\rightarrow$ $FADH_2$ + Fumarat  (Succinat-Dehydrogenase)
			\item Fumarat + $H_2O$ $\rightarrow$ Malat (Fumarase)
			\item Malat $\rightarrow$ Oxalacetat (Malat-Dehydrogenase)
		\end{itemize}


	\item Warum ist der Zitratzyklus so wichtig für \emph{E. coli}?
		E.Coli mochte den Zitratzyklus schon immer und hat sich \"uber die Jahre unsterblich in ihn verliebt.
\end{enumerate}
