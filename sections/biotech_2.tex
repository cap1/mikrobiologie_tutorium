\sectionmark{Biotechnologie II}

\section{Biotechnologie II}
\begin{enumerate}
	\item Wie unterscheiden sich unvollständige Oxidationen von Biotransformationen (Biokonversionen)?
		
		Bei der vollständigen Oxidation wird Sauerstoff komplett zu \ce{CO2} oxidiert.
		Eine unvollständige Oxidattion ist gegeben wenn,
		wenn der Sauerstoff nicht vollständig oxidiert wird,
		sondern das ein Zwischenprodukt das Ziel der Oxidation ist.
		Unvollständige Oxidationen sind meist mit Wachstumsvorgängen assoziert.
		
		Bei einer Biokonversion handelt es sich um einen Co-Metabolismus,
		welche in einer stationären Phase (also nicht während des Wachstums) stattfinden.
		
	\item Nennen sie biotechnologisch relevante Produkte die auf unvollständigen mikrobiellen Oxidationen beruhen.
		
		\begin{itemize}
			\item Acetat
			\item Gluconat
			\item Fumarat
			\item Citrat und andere organische Säuren
			\item Aminosäuren
			\item Alkohole
		\end{itemize}
		
	\item Nennen Sie Charakteristika der Essigsäurebakterien Gluconobacter und Acetobacter.
	
		\begin{itemize}
			\item Gram-negativ
			\item bewegliche Stäbchen
			\item Aerobier
			\item unvollständige Oxidation von Alkoholen und Glucose
		\end{itemize}
			
	\item Was unterscheidet die Unteroxidierer (Suboxidanten) von den Überoxidierern (Peroxidanten)? Nennen sie typische Vertreter beider Gruppen.
	
		\begin{description}
			\item[Peroxidanten] \hfill \\
				Die gebildeten organischen Säuren werden nach dem Verbrauch des Substrats vollständig oxidiert.
				Sie besitzen meist einen vollständigen Tricarbonsäurezyklus (z.B. \emph{Acetobacter}).
			\item[Suboxidanten] \hfill \\
				Die gebildeten organischen Säuren werden nicht verbraucht.
				Sie besitzen keinen Vollständigen Tricarbonsäurezyklus (z.B. \emph{Gulconobacter}).
		\end{description}
		% TCA == Citratzyklus?!
		
	\item Benennen sie die enzymatischen Schritte für die Umsetzung von Ethanol zu Acetat durch Essigsäurebakterien.
		
		\begin{enumerate}[label=\arabic*)]
			\item Ethanol \textrightarrow Acetaldehyd durch \slshape{Ethanoldehydrogenase}
			\item Acetaldehyd \textrightarrow Essigsäure durch \slshape{Aldehyddehydrogenase}
		\end{enumerate}

	\item Durch welche Verfahren wird Essig hergestellt?
		
		\begin{description}
			\item[Orleans-Methode] \hfill \\
				Ursprüngliches Verfahren, erzeugt besonders aromatischen Essig.
				Liegende Fässer mit 200 bis 500 l Inhalt die zu 2/3 gefüllt sind,
				und auf deren Oberfläche sich eine Schicht aus Essigsäurebakterien befindet.
				Wöchentlich werden 10\% des Inhaltes entnommen
				und durch frischen Wein ersetzt.
			\item[Schnellessig-Verfahren] \hfill \\
				Eine Alkoholische Lösung wird durch Bucheholzspäne getröpfelt die in einem Bottichs sind.
				In diesen Bottich wird von unten Luft eingeblasen.
				Dadurch bildet sich ein Biofilm auf der Oberfläche der Buchenspänen.
				Der Prozess kann kontinuierlich 5 bi 30 Jahre lang laufen.
			\item[submerse Fermentation im Acetator] \hfill \\
				Ein Submerse Fermentation die kontinuierlich betrieben wird.
				Ständig wird die gleiche Menge alkoholische Flüssigkeit nach gefüllt,
				wie auch Essig entfernt wird.
				Spezielle Belüftung und Kühlung ist notwendig,
				und die Bakterien müssen durch Filtration entfernt werden.
				Technisches Verfahren mit einr hohen Effiziens von 90-98\% Umwandlung von
				Alkohol in Essig.
		\end{description}

	\item Welche Rolle spielen Essigsäurebakterien bei der Synthese von Ascorbinsäure?	

		In der Reichstein-Grüssner-Synthese von Ascorbinsäure wird \emph{Acetobacter suboxydans} verwendet,
		um die Umsetzung von D-Sorbit in L-Sorbose durchzuführen.
		Dabei endsteht \ce{NADH2}.

	\item Nennen sie Beispiele für durch Stress-induzierte unvollständige Oxidationen.

		In Tabelle \ref{tab:stressigeunvollstdoxidat} sind die zwei Beispiele für unvollständige Oxidationen
		unter Stressbedingungen dargestellt.

		\begin{table}[ht!]
		\begin{center}
		\begin{tabular}{l p{2.8cm} p{5.5cm}} 
		\toprule
		Produkt	&	Organismus	&	Stress-Bedingung\\
		\midrule
		Citrat		&	\emph{Aspergillus niger}	&	Hohe Saccharose Konzentration; pH < 3; geringe \ce{Fe3+}-Konz.;bestimmte \ce{Zn2+}, \ce{Mn2+}, \ce{Cu2+} Konz \\
		L-Glutamat	&	\emph{Corynebacterium glutamicum} 	&	Biotin-Konz. < 5 \begin{math}\mu\end{math};Zugabe von langkettinge Fettsäuren; Zugabe von Penicillin am Produktionsanfang	\\
		\bottomrule
		\end{tabular}
		\caption{Stressinduzierte unvollständige Oxidation zur industriellen Produktionen.}
		\label{tab:stressigeunvollstdoxidat}
		\end{center}
		\end{table}

	\item Unter welchen Bedingungen produziert \emph{Aspergillus niger} Citrat?
		
		Siehe Tabelle \ref{tab:stressigeunvollstdoxidat}.

	\item Was begünstigt die Glutamat-Ausscheidung von \emph{Corynebacterium glutamicum}?
		
		Siehe Tabelle \ref{tab:stressigeunvollstdoxidat}.

	\item Für welche Anwendungsbereiche ist die biotechnologische Produktion von Aminosäuren bedeutsam?

		\begin{table}[h!]
		\begin{center}
		\begin{tabular}{l l} 
		\toprule
			Anwendungsbereich	&	Aminsäuren \\
			\midrule
			Geschmacksverstärker	&	Monosodiumglutamat \hfill (asiatische Küche)\\
			\multirow{2}{*}{}		&	Glycin \hfill (Süßstoff in Soft-Eis)\\
										&	Asparatam + Phenylalanin \hfill (Süßstoff)\\
			Futterzusatz			&	Lys, Thr, Trp \hfill	(Getreide) \\
										&	Met, Lys, Tr \hfill(Soja) \\
			Antioxidatantion		&	Trp, His, Lys, Cys \hfill (Brot) \\
			Infusionen				&	Trp, Ile, Gln, Pro, Tyr \\
			Kosmetika				& 	Ser \\
		\bottomrule
		\end{tabular}
		\label{tab:anwenungAS}
		\caption{Anwendungsbreiche von Aminsäuren.}
		\end{center}
		\end{table}

	\item Wie unterscheiden sich unvollständige Oxidationen und Sekundärstoffwechsel voneinander?
		
		Eine unvollstädnige Oxidation dient in Wachstumsphasen dazu um bestimmte Stoffe für den
		Aufbau der Zelle bereit zustellen.
		Durch einen Sekundärenstoffwechsel,
		besteht ständig Möglichkeit ständig seltener nötige Stoffer herzustellen,
		beispielsweise Antibiotika.
		%ist das so?

	\item Was versteht man unter Tropo- und Idiophase?
		
		\begin{description}
			\item[Tropophase] \hfill \\
				Die Wachstumsphase einer Zelle in der hauptsächlich der Primärmetabolit erzeugt wird.
			\item[Idiophase] \hfill \\
				Die Produktionssphase einer Zelle in der hauptsächlich der Sekundärmetabolit erzeugt wird.
		\end{description}

	\item Nennen Sie Beispiele für den Wirkort von Antibiotika.

		\begin{table}[ht!]
		\begin{center}
		\begin{tabular}{l l} 
		\toprule
			Wirkort	&	Antibiotika \\
			\midrule
			DNS-Replikation		&	Nitroimidazole \\
										&	Fluorcinolone 	\\
			Zellwandbiosynthese	&	\begin{math}\beta\end{math}-Lactame \\
%			\multirow{2}{*}{}	&	Fluorcinolone 	\\
		\bottomrule
		\end{tabular}
		\label{tab:wirkorteantibiose}
		\caption{Angriffsvektoren und die passenden Antiobiotika.}
		\end{center}
		\end{table}

	\item Welche Organismen können Vitamin B12 synthetisieren?
	\item Wie ist ein Metagenom definiert?
	\item Nennen sie zwei grundsätzliche Vorgehensweisen zur Gewinnung von neuen Genen und Biokatalysatoren aus Metagenomen mir Beispielen?
\end{enumerate}
