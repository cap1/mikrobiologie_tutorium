\sectionmark{Biotechnologie II}

\section{Biotechnologie II}
\begin{enumerate}
	\item Wie unterscheiden sich unvollständige Oxidationen von Biotransformationen (Biokonversionen)?
		
		Bei der vollständigen Oxidation wird Sauerstoff komplett zu \ce{CO2} oxidiert.
		Eine unvollständige Oxidattion ist gegeben wenn,
		wenn der Sauerstoff nicht vollständig oxidiert wird,
		sondern das ein Zwischenprodukt das Ziel der Oxidation ist.
		Unvollständige Oxidationen sind meist mit Wachstumsvorgängen assoziert.
		
		Bei einer Biokonversion handelt es sich um einen Co-Metabolismus,
		welche in einer stationären Phase (also nicht während des Wachstums) stattfinden.
		
	\item Nennen sie biotechnologisch relevante Produkte die auf unvollständigen mikrobiellen Oxidationen beruhen.
		
		\begin{itemize}
			\item Acetat
			\item Gluconat
			\item Fumarat
			\item Citrat und andere organische Säuren
			\item Aminosäuren
			\item Alkohole
		\end{itemize}
		
	\item Nennen Sie Charakteristika der Essigsäurebakterien Gluconobacter und Acetobacter.
	
		\begin{itemize}
			\item Gram-negativ
			\item bewegliche Stäbchen
			\item Aerobier
			\item unvollständige Oxidation von Alkoholen und Glucose
		\end{itemize}
			
	\item Was unterscheidet die Unteroxidierer (Suboxidanten) von den Überoxidierern (Peroxidanten)? Nennen sie typische Vertreter beider Gruppen.
	
		\begin{description}
			\item[Peroxidanten] \hfill \\
				Die gebildeten organischen Säuren werden nach dem Verbrauch des Substrats vollständig oxidiert.
				Sie besitzen meist einen vollständigen Tricarbonsäurezyklus (z.B. \emph{Acetobacter}).
			\item[Suboxidanten] \hfill \\
				Die gebildeten organischen Säuren werden nicht verbraucht.
				Sie besitzen keinen Vollständigen Tricarbonsäurezyklus (z.B. \emph{Gulconobacter}).
		\end{description}
		% TCA == Citratzyklus?!
		
	\item Benennen sie die enzymatischen Schritte für die Umsetzung von Ethanol zu Acetat durch Essigsäurebakterien.
	\item Durch welche Verfahren wird Essig hergestellt?
	\item Welche Rolle spielen Essigsäurebakterien bei der Synthese von Ascorbinsäure?
	\item Nennen sie Beispiele für durch Stress-induzierte unvollständige Oxidationen.
	\item Unter welchen Bedingungen produziert \emph{Aspergillus niger} Citrat?
	\item Was begünstigt die Glutamat-Ausscheidung von \emph{Corynebacterium glutamicum}?
	\item Für welche Anwendungsbereiche ist die biotechnologische Produktion von Aminosäuren bedeutsam?
	\item Wie unterscheiden sich unvollständige Oxidationen und Sekundärstoffwechsel voneinander?
	\item Was versteht man unter Tropo- und Idiophase?
	\item Nennen Sie Beispiele für den Wirkort von Antibiotika.
	\item Welche Organismen können Vitamin B12 synthetisieren?
	\item Wie ist ein Metagenom definiert?
	\item Nennen sie zwei grundsätzliche Vorgehensweisen zur Gewinnung von neuen Genen und Biokatalysatoren aus Metagenomen mir Beispielen?
\end{enumerate}
