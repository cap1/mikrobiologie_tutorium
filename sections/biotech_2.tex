\sectionmark{Biotechnologie II}

\section{Biotechnologie II}
\begin{enumerate}
	\item Wie unterscheiden sich unvollständige Oxidationen von Biotransformationen (Biokonversionen)?
	\item Nennen sie biotechnologisch relevante Produkte die auf unvollständigen mikrobiellen Oxidationen beruhen.
	\item Nennen Sie Charakteristika der Essigsäurebakterien Gluconobacter und Acetobacter.
	\item Was unterscheidet die Unteroxidierer (Suboxidanten) von den Überoxidierern (Peroxidanten)? Nennen sie typische Vertreter beider Gruppen.
	\item Benennen sie die enzymatischen Schritte für die Umsetzung von Ethanol zu Acetat durch Essigsäurebakterien.
	\item Durch welche Verfahren wird Essig hergestellt?
	\item Welche Rolle spielen Essigsäurebakterien bei der Synthese von Ascorbinsäure?
	\item Nennen sie Beispiele für durch Stress-induzierte unvollständige Oxidationen.
	\item Unter welchen Bedingungen produziert \emph{Aspergillus niger} Citrat?
	\item Was begünstigt die Glutamat-Ausscheidung von \emph{Corynebacterium glutamicum}?
	\item Für welche Anwendungsbereiche ist die biotechnologische Produktion von Aminosäuren bedeutsam?
	\item Wie unterscheiden sich unvollständige Oxidationen und Sekundärstoffwechsel voneinander?
	\item Was versteht man unter Tropo- und Idiophase?
	\item Nennen Sie Beispiele für den Wirkort von Antibiotika.
	\item Welche Organismen können Vitamin B12 synthetisieren?
	\item Wie ist ein Metagenom definiert?
	\item Nennen sie zwei grundsätzliche Vorgehensweisen zur Gewinnung von neuen Genen und Biokatalysatoren aus Metagenomen mir Beispielen?
\end{enumerate}
