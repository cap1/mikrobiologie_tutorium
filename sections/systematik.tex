\sectionmark{Systematik der Mikroorganismen}

\section{Einführung}
\begin{enumerate}
\item Warum ist die ribosomale RNA so gut als Marker für die Untersuchung von Verwandtschaftsverhältnissen geeignet?
\item Wie geht man bei der Beschreibung eines neuen Organismus vor?
\end{enumerate}

\section{Archaea}
\begin{enumerate}
	\item Vergleichen Sie wichtige molekularbiologische und physiologische Eigenschaften der drei Organismenreiche!
	\item Nennen Sie zwei wichtige Gruppen der Euryarchaeota! Wo würden Sie diese Organismen suchen?
	\item Wiederholen Sie kurz die wichtigsten Schritte der Methanbildung! Welche Rolle spielen ungewöhnliche Koenzyme bei diesem Stoffwechselweg?
	\item Wie schützt Thermoplasma seine Membran vor der hohen Temperatur?
	\item Bei der PCR werden DNA-Fragmente zunächst bei 96°C aufgeschmolzen. Einige Archaea leben bei wesentlich höheren Temperaturen. Daher gibt es zwei Möglichkeiten für deren DNA:
a) Sie haben einzelsträngige DNA. Wie könnte in diesem Fall die Replikation ablaufen?
b) Sie haben doppelsträngige DNA. Wie kann der Doppelstrang bei so hohen Temperaturen stabil sein? Entscheiden Sie sich für eine der Möglichkeiten und beantworten Sie die Zusatzfrage!
\end{enumerate}
	
\section{Bakterien}
\begin{enumerate}
	\item Deinococcus radiodurans kann sehr hohe Strahlendosen überleben. Welche Mechanismen helfen diesem Bakterium dabei?
	\item Zeichnen Sie schematisch den Aufbau einer Spirochaeten-Zelle auf (Längsansicht und Querschnitt). Erklären Sie die wichtigsten Komponenten!
	\item Erläutern Sie den Lebenszyklus der Chlamydien! Warum sind diese Bakterien (C. trachomatis) als Krankheitserreger so bedeutsam?
	\item Die Cyanobakterien sind eine extrem erfolgreiche Bakteriengruppe: Sie leben nicht gerade von Luft und Liebe, aber trotzdem von sehr einfachen Verbindungen. Nennen Sie C-Quelle, N-Quelle, Elektronendonor und Energiequelle dieser Bakterien! 
	\item Glauben Sie, dass ein solch vielfältiges Leben, wie wir es auf der Erde kennen, möglich wäre, wenn es die oxygene Photosynthese nicht geben würde?
	\item Cyanobakterien und -Proteobakterien können leicht mit eukaryontischen Zellen interagieren. Nennen Sie ein paar Beispiele für solche Interaktionen (mindestens sechs)!
	\item Was sind die Schritte, einen künstlichen Organismus herrzustellen? Halten Sie das für eine gute Idee?
	\item Milchsäurebakterien spielen für viele biotechnologische Vorgänge in der Nahrungsmittelindustrie eine große Rolle. Was unterscheidet homo- und heterofermentative Milchsäurebakterien? Sie erhalten die Aufgabe, ein heterofermentatives Bakterium in ein homofermentatives zu verwandeln. Wie gehen Sie vor?
	\item Streptomyceten sind wichtige Antibiotikabildner. Nennen Sie einige von Streptomyceten gebildete Antibiotika und die Erreger, die man damit bekämpfen kann. Warum sterben eigentlich die Streptomyceten nicht selbst an den von ihnen gebildeten Antibiotika?
	\item Wofür steht das E. in \emph{E. coli}? 

		Escheria.
	\item Ralstonia eutropha wurde in Weende entdeckt. Welche Eigenschaften machen dieses Bakterium so interessant?
	\item Zu den Gamma-Proteobakterien gehören die Enterobakterien. Nennen Sie einige Vertreter dieser Bakteriengruppe!
	\item Zu den Delta-Proteobakterien gehören die Gattungen Myxococcus, Bdellovibrio und Geobacter. Nennen Sie die wichtigsten Eigenschaften/ Charakteristika dieser Bakterien!
\end{enumerate}

\section{Viren und Prionen}
\begin{enumerate}
	\item Welche subzellulären Krankheitserreger kennen Sie? Bei welchen Organismengruppen können diese Erreger Krankheiten auslösen?
	\item In welchen Zustandsformen können Viren und Phagen nach der Infektion in der Zelle vorliegen!
	\item Welches genetische Material besitzt das HI-Virus? 
	\item Erläutern Sie kurz den Lebenszyklus des Phagen Lambda!
	\item Bei der spongiformen Enzephalitis kann eine erblich bedingte Krankheit infektiöse Erreger hervorbringen. Wie ist das möglich?

\end{enumerate}
