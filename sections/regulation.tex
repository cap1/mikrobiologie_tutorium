\sectionmark{Regulation}

\section{Regulation}
\begin{enumerate}
	\item In der Glykolyse sind die Phosphofructokinase und die Pyruvatkinase Angriffsorte der allosterischen Regulation. Warum ist das sinnvoll?
	\item Warum ist es so wichtig, dass Mikroorganismen sehr schnell auf äußere Einflüsse reagieren können? Können Sie sich Bakterien vorstellen, die weitgehend ohne regulatorische Vorgänge auskommen können?
	\item Wie kann die Aktivität von Proteinen, z.B. von Enzymen oder Regulatoren moduliert werden? Nennen Sie für wenigstens zwei der Mechanismen Beispiele!
	\item Wodurch zeichnen sich gute Promotoren aus? Tatsächlich sind nur wenige Promotoren wirklich gut! Warum ist es für die Organismen vorteilhaft, suboptimale Promotoren zu besitzen? 
	\item Erläutern Sie den Aufbau der RNA-Polymerase?
	\item Unter welchen Bedingungen wird das lac-Operon von \emph{E. coli} exprimiert? Stellen Sie sich vor, Sie hätten eine Mutante, in der das Gen für die Lac-Permease zerstört  ist. Wie wird das lac-Operon in dieser Mutante exprimiert, wenn Lactose in Medium vorhanden ist?
	\item Bacillus subtilis kann keine Lactose verwerten, dafür aber Fruktose. Was vermuten Sie, unter welchen Bedingungen werden die Gene für die Fruktoseverwertung exprimiert und unter welchen nicht?
	\item  Gene für die Fruktoseverwertung kodieren für eine PTS-Fruktosepermease und für ein Enzym, das Fruktose-1,6-Bisphosphat bildet. Wie sieht dann der weitere Stoffwechsel aus? Die Aktivität welcher Enzyme dieses Stoffwechselweges wird durch Metaboliten reguliert?
	\item Erklären Sie  bitte die Rolle der Adenylatzyklase, des Lac-Repressors und des Crp-Proteins für die Expression des lac-Operons.
	\item Ein Schlüsselfaktor bei der Katabolitenrepression in \emph{E. coli} ist das Crr Protein. Erklären Sie die duale Funktion dieses Proteins!
	\item Wie sind Zweikomponentensysteme aufgebaut und wie werden die Signale in ihnen weitergeleitet?
	\item Was verstehen Sie unter einem RNA-Schalter, was unter einem Riboswitch?
	\item Die Transkription der trp-Operons in \emph{E. coli} und \emph{B. subtilis} wird frühzeitig abgebrochen, wenn Tryptophan im Medium vorhanden ist. Erklären Sie anhand einer Skizze, wie diese Regulation zustande kommt und diskutieren Sie Gemeinsamkeiten und Unterschiede der Regulationsmechanismen in den beiden Bakterien!
\end{enumerate}
