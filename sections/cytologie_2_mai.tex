\sectionmark{Cytologie}

\section{Cytologie}
	\begin{enumerate}
		\item Wie funktioniert in Grundzügen ein Fluoreszenzmikroskop? Welche Unterschiede bestehen zur konventionellen Lichtmikroskopie?				
			Mit Licht in bestimmten Wellenlängen werden Stoffe
			oder auch Proteine (z.B. ``GFP'') im Objekt zum	fluoreszieren angeregt.
			So können bestimmte Strukturen sichtbar gemacht werden.
			Bei der Lichtmikroskopie wird das Objekt mit einem Spektrum des sichtbaren Lichtes
			beleuchtet.

		\item Wie hoch ist die Auflösung eines Lichtmikroskops?
			Kennen Sie eine Formel zur Berechnung der Aufösung (vgl. Mikrobiologisches Grundpraktikum)?

		Die Auflösung eines Lichtmikroskops ist Abhängig von der Wellenlänge des Lichtes.
		Diese liegt bei grünem Licht bei 550 nm und begrenzt die die optische Auflösung.
		Die nummerische Apertur liegt bei der Verwendung von Immersionsöl bei etwa 1,3.
		Somit ist der kleinste Abstand zweier Bildpunkte bei einem 100-fach Objektiv:

		\begin{center}
		\begin{math}
			d_{0} = \dfrac{100}{1,30} = 0,3 \hspace{27mm} |\hspace{3mm} \lambda = 550nm
		\end{math}
		\end{center}


		\item Wie wirkt sich die Größe eines Organismus auf die Stoffwechselleistungen aus? Warum?
			
			Das Verhältnis von Oberfläche durch Volumen eines (kugelförmigen) Organismus,
			nimmt mit zunehmendem Radius ab.
			Dies wird mit der Oberflächenregel beschrieben, welche durch Max Rubner anhand von Säugetieren ermittelt wurde.
			Sie besagt, dass die spezifische Stoffwechselrate für (Stoffverbrauch/kg Körpergewicht)
			mit abnehmender Körpergröße ansteigt.

			Die Oberflächenregel lässt sich auch auf Mikroorganismen übertragen.
			Hier wird der Transport des Substrates über die Membran ins Cytosol zum limitierende Faktor.
			Eine Optimierung wird hier duch geringe Zellgröße und hohe Teilungsrate erreicht.
			Auch spezielle Wuchsformen wie zum Beispiel Stäbchen
			oder Fillamente und andere Zellanhängsel vergrößern die Membranfläche.

		\item Stellen Sie die Eigenschaften von Eu- und Prokaryoten einander gegenüber. Welche Eigenschaften (die Prokaryoten nicht haben) ermöglichen es Eukaryoten, große Zellen zu bilden? Warum sind dennoch einige Prokaryoten groß?
		\item Wie klein kann ein autonom lebender Prokaryot werden? Begründen Sie Ihre Abschätzung.
		\item Was sind Einheitsmembranen (unit membranes)? Wo finden sie sich in der prokaryotischen Zelle?
		\item Wie können Membranen modifiziert werden, um sie untwrschiedlichen Umgebungstemperaturen anzupassen?
		\item Welche Proteine finden Sie typischerweise in Membranen? Welche Funktionen habe sie? Was sind Hopanoide?
		\item Stellen Sie vergleichend die unterschiedlichen Zellwand-Typen von verschiedenen Bacteria und Archaea gegenüber.
		\item Wie ist die Äußere Membran (outer membrane) von Bakterien wie  z. B. \emph{E. coli} gebaut?
		\item Worauf beruht  die Festigkeit der Peptidoglycanstruktur? 
		\item Auf welchem Prinzip beruht die Gram-Färbung? Können Sie mit der Färbung sicher die unterschiedlichen Typen von Zellwänden unterscheiden?
	\end{enumerate}
