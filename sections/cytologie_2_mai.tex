\sectionmark{Cytologie}

\section{Cytologie}
	\begin{enumerate}
		\item Wie funktioniert in Grundzügen ein Fluoreszenzmikroskop? Welche Unterschiede bestehen zur konventionellen Lichtmikroskopie?
		\item Wie hoch ist die Auflösung eines Lichtmikroskops ? Kennen Sie eine Formel zur Berechnung der Aufösung (vgl. Mikrobiologisches Grundpraktikum)?
		\item Wie wirkt sich die Größe eines Organismus auf die Stoffwechselleistungen aus? Warum?
		\item Stellen Sie die Eigenschaften von Eu- und Prokaryoten einander gegenüber. Welche Eigenschaften (die Prokaryoten nicht haben) ermöglichen es Eukaryoten, große Zellen zu bilden? Warum sind dennoch einige Prokaryoten groß?
		\item Wie klein kann ein autonom lebender Prokaryot werden? Begründen Sie Ihre Abschätzung.
		\item Was sind Einheitsmembranen (unit membranes)? Wo finden sie sich in der prokaryotischen Zelle?
		\item Wie können Membranen modifiziert werden, um sie untwrschiedlichen Umgebungstemperaturen anzupassen?
		\item Welche Proteine finden Sie typischerweise in Membranen? Welche Funktionen habe sie? Was sind Hopanoide?
		\item Stellen Sie vergleichend die unterschiedlichen Zellwand-Typen von verschiedenen Bacteria und Archaea gegenüber.
		\item Wie ist die Äußere Membran (outer membrane) von Bakterien wie  z. B. E. coli gebaut?
		\item Worauf beruht  die Festigkeit der Peptidoglycanstruktur? 
		\item Auf welchem Prinzip beruht die Gram-Färbung? Können Sie mit der Färbung sicher die unterschiedlichen Typen von Zellwänden unterscheiden?
	\end{enumerate}
