\sectionmark{Chemolithotrophie}

\section{Chemolithotrophie}
	\begin{enumerate}
		\item Wie konnte Winogradsky zeigen, dass die Schwefelbakterien einen chemolithotrophen Stoffwechsel haben?
		\item Erklären Sie anhand der Redoxpotentiale der Reaktionen, welche besonderen Anforderungen die chemolithotrophe Lebensweise erfordert, die bei heterotrophen Wachstum auf Kohlenhydraten nicht auftreten.
		\item Welche Wege der Oxidation reduzierter Schwefelverbindungen kennen Sie?
		\item Welche Aufgabe haben Nitrifizierer und Nitrosifizierer in der Abwasserbehandlung?
		\item In einem geschichteten Sediment laufen die Prozesse der anaeroben Atmung und der Oxidation reduzierter Verbindungen durch Chemolithotrophe ab. Ordnen Sie die Prozesse den Schichten zu und begründen Sie die Zuordnung!
		\item Nehmen Sie einen Stammbaum der Bacteria und markieren Sie die chemolithotrophen Organismengruppen. 
		\item Welche Gruppen der Bacteria betreiben oxygene bzw. anoxygene Photosynthese?
		\item Nenn Sie die wesentlichen Unterschiede zwischen den Prozessen der oxygenen und anoxygenen Photosynthese!
		\item Wie wird bei der Photosynthese ein Protonengradient erzeugt?
		\item Welche akzessorischen Pigmente kennen Sie? 
		\item In einem Teich sind die unterschiedlichen photosynthetischen Organismen nicht gleichmäßig verteilt. Warum?
		\item Wie wird Kohlendioxid fixiert?
	\end{enumerate}
