\sectionmark{Chemolithotrophie}

\section{Chemolithotrophie}
\begin{enumerate}
	\item Wie konnte Winogradsky zeigen, dass die Schwefelbakterien einen chemolithotrophen Stoffwechsel haben?
		
		Durch Beobachtung an einem Bachlauf mit hohem \ce{H2S}-Gehalt.
		Er stellte fest,
		dass bei höheren Konzentrationen von Schwefelwasserstoff auch mehr Schwefel einlagerten.
		So Schlussfolgertee er,
		dass die Bakterien \ce{H2S} oxidierten und als molkeularen Schwefel ablagerten
		und das Mikroorganismen auch in der Lage sind anorganische Substanzen zu oxidieren.

		Chemolithotrophie wird auch als Chemoautotrophie bezeichnet.

	\item Erklären Sie anhand der Redoxpotentiale der Reaktionen, 
			welche besonderen Anforderungen die chemolithotrophe Lebensweise erfordert, 
			die bei heterotrophen Wachstum auf Kohlenhydraten nicht auftreten?

		\begin{equation}
			2\ \ce{H} + \ce{O2} \rightarrow 2\ \ce{H2O} \hspace{1.8cm} \Delta G^0 = -278 kJ
			\label{Hydrolyse}
		\end{equation}

		Die in Formel \ref{Hydrolyse} beschriebene Redox-Reaktion erzeugt.
		\ce{NADH} wird bei Chemolithotrophen Bakterien die diese Reation betreiben
		auch mit einer nicht Membran gebundenen Hydrogenase regeregeneriert.


	\item Welche Wege der Oxidation reduzierter Schwefelverbindungen kennen Sie?
		
		\begin{equation}
			\ce{H2S} + 2\ \ce{H2O} \rightarrow \ce{SO4}^{2-} + 2\ \ce{H+}
			\label{schwefelred}
		\end{equation}

		In Formel \ref{schwefelred} ist der Grundlegende Reaktion dargestellt.
		Diese Reaktion kann über verschiedene Wege realisiert sein.
		Für die Substratketten-Phosphorylierung und den Sulfit-Oxidase-Weg wird der Schwefel-Wasserstoff
		zunächst in \ce{SO3}$^{-2}$ überführt.
		Ausgangspunkte können deshalb auch elementarer Schwefel oder Thiosulfat (\ce{S2O3}$^{-2})$) sein.
		Die dabei endstehenden Elteronen können über das Elektronen-Transport-System weiter verwendet werden.

		\begin{description}
			\item[SOX-Weg] \hfill \\
				Neben der Sulfit-Oxidase-abhängigen Reaktion,
				kann Sulfid von vielen phototrophen Schwefelbakterien (z.B. \emph{Paracoccus pantotrophus})
				direkt über das sogenannte Sox-System oxidiert werden.
				Dieses System stellt die nötige Reduktionskraft zur Kohlenstoffdioxid-Fixierung bereit.
			\item[Substratketten-Phosphorylierung (APS-Weg)] \hfill \\
				Die Umsetzung von Sulfit zu Sulfat kann bei einigen \emph{Thiobacilli} mit einer
				Substratkettenphosphrylierung erfolgen.
				Dabei wird von einer Adenosinphosphosulfat-Reductase (APS-Rductase) AMP zu ADP umgesetzt.
				Eine Kinase erzeugt aus dem ADP wiederum ATP.

				Die APS-Reductase arbeitet gegenläufig der dissimilatorischen Sulfatredukion,
				siehe Abschnitt \ref{sec:anaerobeAtmung} "Anaerobe Atmung".
				
			\item[Sulfit-Oxidase-Weg] \hfill \\
				Unter Abspaltung von zwei Elektronen von Sulfit,
				endsteht Sulfat.
				\begin{equation}
					Sulfit-Oxidase(\ce{SO3}^{2-}) \rightarrow \ce{SO4}^{2-} + 2 e^-
					\label{sulfitoxidase}
				\end{equation}
		\end{description}

	\item Welche Aufgabe haben Nitrifizierer und Nitrosifizierer in der Abwasserbehandlung?
		
		Die Umsetzung von Ammonium (\ce{NH3}) in Nitrat und Nitrit.
		Dieser Vorgang lässt sich zweiteilen (Nitrifizierer/Nitrosifizierer). %TODO Nachlesen

		Der Abbau bestimmter Schwefelhaltiger Substanzen ist zu dem erwünscht,
		da diese zum Teil toxische sind (\ce{H2S}).

	\item In einem geschichteten Sediment laufen die Prozesse der anaeroben Atmung und der
			Oxidation reduzierter Verbindungen durch Chemolithotrophe ab. Ordnen Sie die
			Prozesse den Schichten zu und begründen Sie die Zuordnung!


	\item Nehmen Sie einen Stammbaum der Bacteria und markieren Sie die chemolithotrophen Organismengruppen. 
	\item Welche Gruppen der Bacteria betreiben oxygene bzw. anoxygene Photosynthese?
	\item Nenn Sie die wesentlichen Unterschiede zwischen den Prozessen der oxygenen und anoxygenen Photosynthese!
	\item Wie wird bei der Photosynthese ein Protonengradient erzeugt?
	\item Welche akzessorischen Pigmente kennen Sie? 
	\item In einem Teich sind die unterschiedlichen photosynthetischen Organismen nicht gleichmäßig verteilt. Warum?
	\item Wie wird Kohlendioxid fixiert?
\end{enumerate}
