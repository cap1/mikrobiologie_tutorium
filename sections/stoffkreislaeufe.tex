\sectionmark{Stoffkreisläufe}

\section{Stoffkreisläufe}
\begin{enumerate}
	\item Nennen sie dissimilatorische Prozesse im Stickstoff-Kreislauf?
		
		\begin{itemize}
			\item Anaerobe Denitrifikation \hfill \ce{NO3-} \textrightarrow \ce{NO2-} \textrightarrow \ce{N2}
			\item Aerobe Denitrifikation \hfill \ce{NO4+} \textrightarrow \ce{NO2-} \textrightarrow \ce{NO3-}
			\item Anammox \hfill \ce{NO2-},\ce{NH4+} \textrightarrow \ce{N2}
			\item Assimilation \hfill \ce{NH4+} \textrightarrow \ce{NH3}
		\end{itemize}

	\item Welche Rolle spielen Denitrifikanten und Nitrifikanten bei der Abwasserbehandlung in Kläranlagen?

		Durch Denitrifikation geschieht die Stickstoff-Eleminierung.
		Dies kann auch durch das Anammox-Verfahren geschehen,
		bei dem jedoch weniger Biomasse endsteht.

		Bei der Denitrifikation wir zunächst aerob \ce{NH4+} in \ce{N03-} unter \ce{o2}-Gabe umgesetzt.
		Dann wird das \ce{NO3-} mit organimschem Kohlenstoff,
		beispielsweise aus Methanol,
		in Stickstoffgas und Biomase umgesetzt.

		Beim Aanmmox-Verfharen wird  die Hälfte des \ce{NH4+} mit \ce{O2} in\ce{NO2-} umgestetzt.
		Dieses wird zusammen mit dem restlichen \ce{NH4+} in Stickstoffgas und wenig Biomase umgesetzt.

	\item Welche Umsetzung katalysieren Anammox-Bakterien?
	\item Nennen Sie \ce{N2}-Fixierer!
	\item Können eukaryontische Organismen \ce{N2} fixieren?
	\item Wie ist Nitrogenase aufgebaut?
	\item Beschreiben Sie den Ablauf der \ce{N2}-Reduktion an der Nitrogenase!
	\item Wie kann die Nitrogenase vor dem Kontakt mit Sauerstoff geschützt werden?
	\item Welche Rolle spielen phototrophe Bakterien im Schwefel-Kreislauf?
	\item An welchen Schritten im Stickstoff- und Schwefel-Kreislauf sind chemolithoautotrophe Organismen beteiligt (Nennen sie Namen für typische Vertreter)?
	\item Welche Organismengruppen können für Gebäudeschäden verantwortlich sein und warum?
	\item Welche Rolle spielen Sulfatreduzierer im Schwefel-Kreislauf?
\end{enumerate}
