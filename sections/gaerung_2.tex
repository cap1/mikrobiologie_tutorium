\sectionmark{Gärung II}

\section{Gärung II}
\begin{enumerate}
	\item Was für eine Reaktion katalysiert die Phosphoketolase und für welchen Typ von Milchsäurebakterien ist sie bedeutsam?
	\item Wobei und welche schädlichen reaktiven Sauerstoffverbindungen können entstehen?
	\item Nennen sie Beispiele für Entgiftungsmechanismen.
	\item Nennen sie charakteristische Eigenschaften von Clostridien.
	\item Welche Substrate können saccharolytische bzw. peptolytische Clostridien verwerten?
	\item Welche Gärprodukte können aus Glukose gebildet werden?
	\item Welches charakteristisches Enzym ist bei den Clostridien für die Pyruvatspaltung verantwortlich?
	\item Welchen Vorteil bringt den Clostridien die H2-Entwicklung?
	\item In welchen Reaktionen kann bei Clostridien durch Substratkettenphosphorylierung ATP gebildet werden?
	\item Durch welche Milieubedingungen wird die Bildung von Aceton/Butanol durch Clostridium acetobutylicum begünstigt?
	\item Welcher Zelldifferenzierungsprozess ist mit der Lösungsmittelbildung gekoppelt?
	\item Welche Umstände begrenzen die biotechnologische Produktion von Lösungsmitteln aus Glucose durch \emph{Clostridium acetobutylicum}?
	\item Nennen Sie pathogene Clostridienarten und die pathogene Wirkung.
	\item Was versteht man unter der Stickland-Reaktion? Nennen sie ein Beispiel für diese Reaktion.
	\item Wie kann Ethanol vergärt werden?
\end{enumerate}
