\sectionmark{Gärung II}

\section{Gärung II}
\begin{enumerate}
	\item Was für eine Reaktion katalysiert die Phosphoketolase und für welchen Typ von Milchsäurebakterien ist sie bedeutsam?

		Die Phosphoketolase kataklysiert die heterofermentative Milchsäuregärung.
		Dieser Typ der Gärung wird von Bifidobakterien verwendet-
		Sie sind nicht aerotolerant und benötigen eine \ce{CO2}-Atmosphäre.
		Die heterofermentative Milchsäuregärung (Pentose-P-Weg= wird auch durch geführt von \emph{Leuconostoc} und
		\emph{Lactobacillus brevis}.
		
	\item Wobei und welche schädlichen reaktiven Sauerstoffverbindungen können entstehen?

		\begin{description}
			\item[Superoxid \ce{O2-}] \hfill\\
				Superoxid-Dismutase
			\item[Wasserstoffperoxid \ce{H2O2}] \hfill\\
				Katalase
			\item[Hydroxyl-Radikal \ce{OH-}] \hfill\\
		\end{description}

	\item Nennen sie Beispiele für Entgiftungsmechanismen.
		
			s.o.

	\item Nennen sie charakteristische Eigenschaften von Clostridien.
		
		Die Clostrieden spalten sich auf in die Gruppe der Saccharolytischen Clostrieden,
		welche Zucker verstoffwechseln und
		in die Peptolytischen Clostriedien,
		welche Aminosäuren verstoffwechseln.
		Beiden Gruppen sind folgende Eigenschaften gemein:
		\begin{itemize}
			\item Stäbchenform
			\item Gram-positiv
			\item niediger GC-Gehalt des Genoms
			\item strikt anaerob
			\item Sporenbildner
			\item bevorzugen neutralen oder alkalischen pH
			\item z.T. \ce{N2}-Fixierer
		\end{itemize}

	\item Welche Substrate können saccharolytische bzw. peptolytische Clostridien verwerten?

		\begin{description}
			\item[Saccharolytische Clostridien] \hfill \\
				Cellulose, Zucker, Stärke, Pectin
			\item[Peptolytische Clostridien] \hfill \\
				Aminosäuren
		\end{description}

	\item Welche Gärprodukte können aus Glukose gebildet werden?
		
		Glukose kann von Clostriedien zu Folgenden Substanzen fermentiert werden:
		\begin{itemize}
			\item Acetat
			\item Aceton
			\item Butanol
			\item Butyrat
			\item \ce{CO2}, \ce{H2}
			\item Ethanol
			\item Propionat
			\item Succinat
		\end{itemize}

	\item Welches charakteristisches Enzym ist bei den Clostridien für die Pyruvatspaltung verantwortlich?

		Pyruvat-Ferredoxin Oxidoreduktase.

	\item Welchen Vorteil bringt den Clostridien die \ce{H2}-Entwicklung?
	\item In welchen Reaktionen kann bei Clostridien durch Substratkettenphosphorylierung ATP gebildet werden?

		Bei der Umsetzung von Butyryl-CoA mit der Butyrat-Kinase zu Butyrat und
		bei der Umsetzung von Acetyl~P zu Acetat durch die Acetat-Kinase

	\item Durch welche Milieubedingungen wird die Bildung von Aceton bzw. Butanol durch Clostridium acetobutylicum begünstigt?

		Die Bildung von Aceton und Butanol findet bevorzugt bei leicht sauren pH-Werten statt.
		% wenn richtig interpretiert Gaerung2 S.11 

	\item Welcher Zelldifferenzierungsprozess ist mit der Lösungsmittelbildung gekoppelt?
		
		Der Beginn der Sporenbildung ist mit der Bildung von Butanol, Aceton und Ethanol assoziert.

	\item Welche Umstände begrenzen die biotechnologische Produktion von Lösungsmitteln aus Glucose durch \emph{Clostridium acetobutylicum}?
			
		Butanol ist toxisch für \emph{Clostridium acetobutylicum} und
		begrenzt somit die Konzentration auf etwa 20 g/l.
		Zusätzlich ist auf Grund der Stoffwechselwege 
		der Ertrag auf maximal 30 kg Lösungsmittel pro 100 kg Substrat begrenzt.
		
	\item Nennen Sie pathogene Clostridienarten und die pathogene Wirkung.
		
		\begin{table}[h!]
		\begin{center}
		\begin{tabular}{p{2.399cm} p{3.4cm} p{5.0cm}} 
		\toprule
		Art	&	Verantwortlichkeit	&	Symptome\\
		\midrule
		\emph{C. histolyticum},	\emph{C. septicum}	&	Wundinfektion				&	Wachstum in tiefen Wunde, übelriechend \\
		\emph{C. tetani}				&	Wundstarrkrampf			&	Tetanus-Toxin: blockiert Neurotransmitterausschüttung  an inhibitorischen Synapsen \\
		\emph{C. botulinum}			&	Lebensmittelvergiftung, Botulismus	&	Tetanus-Toxin: blockiert Neurotransmitterausschüttung an Synapsen der stimulatorischen Neuronen an Muskeln \\
		\emph{C. butyricum}, \emph{C. sporogenes}			&	Lebensmittelverderb		&	Verderb unzureichen haltbargemacher Lebensmittel, Explodierende-Konserven\\
		\bottomrule
		\end{tabular}
		\caption{Verschiedene pathogene Clostriedien und die von ihen Erzeugten Krankheiten.}
		\label{tab:pathClostridien}
		\end{center}
		\end{table}

	\item Was versteht man unter der Stickland-Reaktion? Nennen sie ein Beispiel für diese Reaktion.
		
		Die Stickland-Reaktion ist die paarweise Vergärung von Aminosäuren.
		Es handelt sich hierbei um eine Redox-Reaktion bei der eine Aminosäure oxidiert,
		die andere reduziert wird.
		Die Produkte einer solchen Reaktion sind iummer \ce{NH3}, \ce{CO2} und 
		eine Carbonsäure mit einem C-Atom weniger als die oxidierte Aminosäure.

		In Tabelle \ref{tab:stickland} befindet sich eine Aufstellung von möglichen Aminsäurepaaren.
		
		\begin{table}[h!]
		\begin{center}
		\begin{tabular}{l l} 
		\toprule
		oxidierte Aminosäure	&	reduzierte Aminosäure	\\
		\midrule
		Alanin		&	Glycin			\\
		Leucin		&	Proline			\\
		Isoleucin	&	Hydroxyprolin	\\
		Valin			&	Tryptophan		\\
		Histidin		&	Argenin			\\
		\bottomrule
		\end{tabular}
		\caption{Paare von Aminosäuren die eine Stickland-Reaktion durchlaufen können.}
		\label{tab:stickland}
		\end{center}
		\end{table}

	\item Wie kann Ethanol vergärt werden?
		
		Ethanol kann mit Wasser zu Acetat und Wasserstoff fermentieren.
		
		\ce{2CH3CH2OH} + \ce{2H2O} \textrightarrow \ce{4H2} + \ce{2CH3COO-} + \ce{2H+}
\end{enumerate}
