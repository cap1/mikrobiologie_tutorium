\sectionmark{Wachstum der Bakterien II}

\section{Wachstum der Bakterien II}
\label{sec:wachstum2}
\begin{enumerate}
	\item Viele Archaea leben bei extrem hohen Temperaturen, bei denen der DNS-Doppelstrang schmilzt. Haben diese Organismen eine einzel- oder doppelsträngige DNS? Wenn sie doppelsrtängig ist, wie wird erreicht, dass die Stränge nicht auseinanderfallen?
		\label{item:htarchaea}
		Hyperthermophile Archaeen besitzen spezielle Anpassung an extrem hohe Temperaturen.
		Zu dieses speziellen Anpassung zählt ein geringe GC-Gehalt des Genoms,
		da diese Bindung weniger stabil ist.
		Die DNS ist weiterhin mit Proteinen assoziert,
		dies können z.B. Histone sein.
		Durch positives Supercoiling wird die DNS besser vor Temperatureinflüssen geschützt.
		Die reverse Gyrase, welche für diesen Vorgang benötigt wird,
		ist das einzige für hyperthermophile Bakterien und Archaeen spezifische Protein.
		Die Erhöhung der intrazellulären Salzkonzentration verhindert ebenso DNS-Schäden.

		Weiter Anapssungen an hohe Temperaturen umfassen vorallem die Membranen.
		So werden zum Teil die Esterbindungen zwischen Glycerol und den Seitenketten durch Etherbindungen ersetzt.
		Dadurch ergeben sich weniger Doppelbindungen,
		was siche postiv auf die Stabilität der Membran auswirkt.
		Bei eingen Arten wird die Lipid-Doppelschicht durch einen Biphyntanyl-Monolayer erstzt.

		Als Beispiel für ein hyperthermophilen Mikroorganismus ist hier \emph{Pyrolobus fumarii} aufgeführt.
		In Tabelle \ref{tab:pfumariikardinal} sind sine Kardinaltemperaturen aufgeführt 
		(Vergleiche: Tabelle \ref{tab:ecolikardinal}, ``Kardinaltemperaturen von \emph{E. coli}).

		\begin{table}[h]
		\begin{center}
		\begin{tabular}{l r}
		\toprule
		Kardinaltemperatur	&	Wert	\\
		\midrule
		Minimaltemperatur		&	90\textdegree C		\\
		Optimaltemperatur		&	106\textdegree C	\\
		Maximaltemperatur		&	113\textdegree C	\\
		\bottomrule
		\end{tabular}
		\caption{Kardinaltemperaturen von \emph{Pyrolobus fumarii}}
		\label{tab:pfumariikardinal}
		\end{center}
		\end{table}

	\item Warum ist \emph{Prochlorococcus marinus} von so großer Bedeutung? Wo leben diese Bakterien?

		\emph{P. marinus} ist wohl der häufigst und am weitesten verbreitet Organismus der Welt.
		Weiterhin gehört es zu den kleinsten bekannten Cyanobakterien, 
		und lebt in den Ozeanen von 40\textdegree Nord bis 40\textdegree Süd in Meerestiefen von 100 bis 200m.
		Durch die Konzentration von 10\textsuperscript{4} bis 10\textsuperscript{5} Zellen pro ml Meerwasser,
		ist er wahrscheinlich der wichtigste Primäproduzent im Ozean.
		Einzelne Zellen von \emph{P. marinus} haben einen Durchmesser von 0,5 bis 0,8 \begin{math}\mu m\end{math}.

	\item Welche Anpassungen weisen die Lipide von hyperthermophilen Organismen auf?

		Siehe Frage \ref{sec:wachstum2}.\ref{item:htparchaea}.

	\item Welche Bakteriengruppen spielen in der Darmflora eine besonders wichtige Rolle?
		
		In Tabelle \ref{tab:darmMO} befindet eine Übersicht über das Microbiom im menschlichen Darm.

		\begin{table}[h]
		\begin{center}
		\begin{tabular}{l l}
		\toprule
		Domäne		&		Verbreitung \\
		\midrule
		Archaea		&		\emph{Methanobervibacterium smithii}	\\
		Eukarya		&		wenige Pilze und Hefen 		\\
		Bacteria		&		Vertreter aus 9 Phyla, circa 1000 Arten 	\\
		Phagen		&		1200 virale Genotypen, meist Phagen von Firmicuten \\
		\bottomrule
		\end{tabular}
		\caption{Übersicht über (bekannte) Mikroorganismen im menschlichen Darm}
		\label{tab:darmMO}
		\end{center}
		\end{table}

		Die Darmflora setzt sich zum 99\% aus den bakteriellen Gruppen \emph{Firmicutes},
		\emph{Bacteroidetes}, \emph{Proteobacteria} und \emph{Actinobacteria} zusammen.		% cc-sa by de.wikipedia.org/wiki/Darmflora

	\item Welche Leistung erbringt die Darmflora für den Menschen?
		
		Die Darmflora erbringt zahlreiche Leistungen für den Menschen.
		Durch die starke Integration und Interaktion der Mikroorganismen mit dem Menschen,
		wir dieser auch als ``Superorganismus'' bezeichnet.

		\begin{description}
		\item[Häufige Gene im Microbiom] \hfill \\
			Depolymerasen im Kohlenstoff-Stoffwechsel
		\item[Nahrungskette] \hfill
			\begin{itemize}
			\item Auf - und Abbau von Nahrung: \\
			Polysaccharide \textrightarrow Karbonsäuren \textrightarrow Gase
			\item Methanbildung zur Entfernung von Wasserstoff
			\end{itemize}
		\item[Organische Säuren] \hfill \\
			Aufnahme durch den Wirt - bis zu 10\% der Kalorienaufnahme des Wirtes	
		\end{description}

		Die Zusammensetzung des Mikrobioms ist abhängig von der Ernährung des Menschen.
		Schlanke Menschen haben einen erhöhten Anteil von bis zu 20\% \emph{Bacteroides},
		wohingengen Vollschlanke eine höheren Anteil an \emph{Firmicutes} im Microbiom haben.
		Jedoch ist die Zusammensetzung des Microbioms eine folge des Köpergewichtes,
		nicht umgekehrt.

\end{enumerate}
