\sectionmark{Aerober Stoffwechseli II}

\section{Aerober Stoffwechsel II}
\begin{enumerate}
	\item Was ist die protonenmotorische Kraft? Welche Möglichkeiten zu Ihrer Gewinnung kennen Sie?
	\item Wozu benutzen Bakterien die protonenmotorische Kraft?
	\item Durch welche Energiequellen kann die Substrataufnahme angetrieben warden?
	\item Erklären Sie die Funktionsweise des Phosphotransferasesystems!
	\item Sind Bakterien lebensfähig, wenn die protonenmotorische Kraft zusammenbricht? 
	\item Erläutern Sie die Funktionsweise der  ATPase! Wozu wird die Energie aus der PMF bei der ATP-Synthese benötigt?
	\item Benötigen anaerobe Bakterien eine PMF? Wenn ja, welche Möglichkeiten zu ihrem Aufbau haben sie?
	\item Welche Klassen von Elektronenüberträgern bei der Atmung kennen Sie? Wie sind diese Moleküle prinzipiell aufgebaut?
	\item Erklären Sie  den Aufbau der Cytochrome? Was ist ein Porphyrin? Kennen Sie neben dem Häm noch weitere Porphyrine und bei  welchen Funktionen spielen sie eine Rolle? Welche Metallionen enthalten sie? (Denken Sie dabei z. B. an das  Coenzym F430)!
	\item Wichtige Proteine der Atmungskette werden als Dehydrogenasen, Oxidasen und Reduktasen bezeichnet. Was verstehen Sie darunter? Welche Funktion hat jedes dieser Enzyme?
	\item Die Glykolyse und der Citratzyklus sind die zentralen Drehscheiben des Stoffwechsels. Welche Metaboliten werden aus Intermediaten dieser Wege hergestellt?
\end{enumerate}
