\sectionmark{Aerober Stoffwechseli II}

\section{Aerober Stoffwechsel II}
\begin{enumerate}
	\item Was ist die protonenmotorische Kraft? Welche Möglichkeiten zu Ihrer Gewinnung kennen Sie?
		Potentielle Energie im elektrochemischen Protonengradienten \\
		Wirkung der PMF: zieht Protonen ins Zellinnere \\
		Gewinnung:  \\
		\begin{itemize}
			\item Wichtigste Quelle: Redoxreaktionen in Membranen
			\item ATP-Hydrolyse bei G\"arern
			\item An $Na^+$-Pumpen gekoppelte S\"aure Dekarboxylasen	
		\end{itemize}

	\item Wozu benutzen Bakterien die protonenmotorische Kraft?
		Ich denk mal Atmung (ATP-Synthese) 
	\item Durch welche Energiequellen kann die Substrataufnahme angetrieben warden?
		?	
	\item Erklären Sie die Funktionsweise des Phosphotransferasesystems! Das PTS ist ein aktives Stofftransportsystem. Dabei: \"Ubertragen des energetischen PEP Phosphatrests auf das Substrat (Meist Hexosen oder Zuckeralkohole). Das Suobstrat kann un durch die beteiligten Membranproteine ins Cytoplasma eingeschleust werden.

	\item Sind Bakterien lebensfähig, wenn die protonenmotorische Kraft zusammenbricht? Bakterien, die Energie auch ohne aeroben Stoffwechsel gewinnen k\"onnen ja, andere nicht. (Not sure 'bout that). 
		
	\item Erläutern Sie die Funktionsweise der  ATPase! Wozu wird die Energie aus der PMF bei der ATP-Synthese benötigt?
		Die ATP-Asen sind Trasmembranproteine, die entweder als ATP-Verbrauchende Protonenpumpe oder als Protonen-getriebene ATP-Synthese.
		Der Enzymkomplex ist ein molekularer Motor aus zwei Teilen: einer rotierenden und einer stationaeren Komponente. Die Rotation der $\gamma$-Untereinheit induziert Strukturver\"anderungen in der $\beta$-Einheit, die zur Synthese und Freisetzung des ATP vom Enzym f\"uhren. Der Protonenfluss druch die ATP-Synthase liefert die Kraft f\"ur die Rotation.
	\item Benötigen anaerobe Bakterien eine PMF? Wenn ja, welche Möglichkeiten zu ihrem Aufbau haben sie?	\\ 
		M\"oglichkeiten zum Aufbau: S\"aure Dekarboxylasen gekoppelt an $Na^+$-Pumpen, Symport von Fermentationsprodukten und Protonen \\
		Eine PMF kann neben der ATP-Gewinnung auch zur Geißelbewegung oder W\"armeerzeugung benutzt werden.
	\item Welche Klassen von Elektronenüberträgern bei der Atmung kennen Sie? Wie sind diese Moleküle prinzipiell aufgebaut?
		\begin{itemize} 
			\item Flavoproteine: Prosthetische Gruppe: Flavin-Nukleotide (z.B. FAD)
			\item Chinone: stark hydrophode Isoprenoid-Seitenketten, 3 Typen: 
				\begin{itemize} 
					\item Ubichinon
					\item Menachinon (Vitamin K Derivat)
					\item Plastochinon
				\end{itemize}

			\item Fe-S Proteine: Eisen-Schwefel-Cluster, ($Fe_2S_2$, $Fe_4S_4$) 
			\item Cytochrome: Proteine mit z.B.  Haem als prosthetische Gruppe (Haem: 4 Pyrrolringe = Tetrapyrol = Porphyrinring)

		\end{itemize}
	\item Erklären Sie  den Aufbau der Cytochrome? Was ist ein Porphyrin? Kennen Sie neben dem Häm noch weitere Porphyrine und bei  welchen Funktionen spielen sie eine Rolle? Welche Metallionen enthalten sie? (Denken Sie dabei z. B. an das  Coenzym F430)! \\
		Proteine mit z.B. H\"am ls prosthetische Gruppe (Haem: 4 Pyrrolringe = Tetrapyrol = Porphyrinring)
		In der Mitte des Porphyrinrings ist ein Metallion eingeschlossen. M\"ogliche Ionen sind z.B. Eisen, Magnesium, Nickel.
			

	\item Wichtige Proteine der Atmungskette werden als Dehydrogenasen, Oxidasen und Reduktasen bezeichnet. Was verstehen Sie darunter? Welche Funktion hat jedes dieser Enzyme?
		\begin{itemize}
			\item Dehydrogenasen: \"Ubernehmen Elektronen von Elektronendonor
			\item Oxidasen: reduzieren Terminalen Elektronenakzeptor Sauerstoff
			\item Reduktasen: reduzieren alternative terminale Elektronenakzeptoren
		\end{itemize}
	\item Die Glykolyse und der Citratzyklus sind die zentralen Drehscheiben des Stoffwechsels. Welche Metaboliten werden aus Intermediaten dieser Wege hergestellt?
		\begin{itemize}
			\item Fetts\"auren
			\item Porphyrine, H\"aeme
			\item Aminos\"auren
			\item Pyrimidine
			\item Wenn ben\"otigt auch Glukose (Glukoneogenese)
		\end{itemize}
\end{enumerate}
