\sectionmark{Fragenkatalog 1-3}

\section{Einführung ``Mikrobenparade''}
	\begin{enumerate}
		\item Welches sind die Merkmale des Lebens? Warum sind Prionen keine Lebewesen? Warum können sie trotzdem infektiös sein? \hfill \vspace{4mm}
		Merkmale des Lebens sind zum Beispiel ein Stoffwechsel \& Fortpflanzung.
		Prionen haben keinen eigenen Stoffwechsel,
		die ``chemischen'' Fähigkeiten der Proteine reichen für das Übernehmen von fremden Zellen aus.
		Durch die Übernahme der Zellen sind sie mit ihrem eigenen Erbgut in der Lage,
		sich zu vermehren - deshallb sind sie infektiös.

	\item Nennen Sie mindestens fünf durch Bakterien verursachte Krankheitserreger, die Namen der entsprechenden Bakterien sowie die Bakteriengruppen, denen sie zuzuordnen sind!
		\begin{enumerate}
			\item \emph{Yersinia pestis} \hfill \\
				Pest, Gammaproteobacteria
			\item \emph{Vibrio cholerae} \hfill \\
				Cholera, Proteobacteria
			\item \emph{Bordetella pertussis} \hfill \\
				Keuchhusten, Proteobacteria
			\item \emph{Mycoplasma pneumoniae} \hfill \\
				Lungenentzündung, Firmicutes
			\item \emph{Bacillus anthracis} \hfill \\
				Milzbrand, Clostridien
		\end{enumerate}

	\item Welche Rolle spielen die Plasmide für die Virulenz von \emph{Bacillus anthracis}?  \hfill \vspace{0.2mm} \\
			Die Virulenz von \emph{Bacillus anthracis} ergibt sich
			aus den Plasmiden ``px01'' und ``px02''.
			Im ersten Plasmid (``px01'') wir ein potentes Toxin codiert,
			das zweite Plasmid (``px02'') ist für die Bildung einer Kapsel verantwortlich.
			Durch die Kapsel wird die Phagocytose der Zelle verhindert,
			wodurch das Toxin wirken kann.

		\item Nennen Sie einige Bakterien, die für die Landwirtschaft/unsere Ernährung von entscheidender Bedeutung sind. \hfill \\
	\emph{Saccharomyces cerevisae}, die Bier- \& Bäckerhefe,
	ist ein wichtiger Teil der Nahrungsmittelproduktion.
	\emph{Lactobacillus casei} ermöglicht die zum Beispiel die Herstellung von Sauerkraut,
	und das Haltbarmachen von Milchprodukten.
	Diverse Arten von \emph{Rhizobium} ermögliche die Fixierung von molekularen Stickstoff,
	den sie den Wurzeln von Pflanzen bereitstellen.
	Diese, sogenannten ``Knölchenbakterien'', sind Grundlesgen für die Stickstoffversorgung der Pflanzen.
	\end{enumerate}


\section{Geschichte der Mikrobiologie}
	\begin{enumerate}
		\item Womit beschäftigen sich die Koch’schen Postulate? Bei welcher Gelegenheit wurden Sie von Robert Koch aufgestellt und was besagen sie? \hfill \vspace{4mm}
		\item In einem klassischen Experiment konnte Griffith 1928 einen apathogenen Stamm von Streptococcus pneumoniae in einen pathogenen Stamm umwandeln.\\
			Wie funktionierte dieses Experiment? Worauf beruht es? Welche bahnbrechende Schlußfolgerung wurde daraus gezogen? \hfill \vspace{4mm}
		\item Mycoplasmen sind die wichtigsten Objekte der synthetischen Biologie. Nennen Sie einige Eigenschaften dieser Bakterien, die sie für die Forschung allgemein und die synthetische Biologie im besonderen interessant machen. \hfill \vspace{4mm}
	\end{enumerate}

