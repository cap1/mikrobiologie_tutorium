\sectionmark{Gärung I}

\section{Gärung I}
\begin{enumerate}
	\item Was versteht man unter einer klassischen Gärung und wie unterscheidet sie sich von Atmungsprozessen?
	
		Als Gärung bezeichnet man eine intern ausbalancieten Oxidations-Reduktion.
		Die klassischen Charakteristiken einer Gärung sind:
		\begin{itemize}
			\item keine externen e\textsuperscript{-}-Akzeptoren
			\item keine direkte mit Substratoxidation gekoppelte Elektronentransport-Phosphorylierung.
			\item organische Verbindungen als Substrate
		\end{itemize}

	\item Benennen Sie Gärungsprodukte und Organismen, die diese Produkte produzieren.
		
		In Tabelle \ref{tab:gaerungsprodukte} sind einige Organismen und ihre Gärungsprodukte aufgeführt.
		Zentralles Intermediat ist in den aufgeführten Vorgängen immer Pyruvat.
		\begin{table}[h!]
		\begin{center}
		\begin{tabular}{l l} 
		\toprule
			Organismengruppe			&	Gärungsprodukt\\
			\midrule
			Milchsäurebakterien		&	Lactat\\
			Hefen							&	Ethanol\\
			Propionibakterien			&	Propionat\\
			Coli-Aerogenes-Gruppe	&	Butandiol\\
			Clostridien					&	2-Propanol, Butanol\\
		\bottomrule
		\end{tabular}
		\caption{Verschiedenen Mikroorganismen und ihre Gärungsprodukte.}
		\label{tab:gaerungsprodukte}
		\end{center}
		\end{table}

	\item Wieso muss bei einer Gärung die Wasserstoffbilanz ausgeglichen werden?
	\item Auf welchen Wegen kann der initiale Glucoseabbau bei unterschiedlichen Gärungen erfolgen (mit Beispielen)?
	\item Geben Sie Beispiele für C-Quellen, die von Saccharomyces fermentiert, veratmet, bzw. nicht verwertet werden können.

	Glucose (\ce{C6H12O6}) kann sowohl veratmet als auch vergärt werden.	
		\begin{description}
			\item[Atmung] \hfill\\
				\ce{C6H12O6} + \ce{602} \textrightarrow \ \ce{6CO2} + \ce{6H2O} \hfill 32-36 ATP	
			\item[Gärung] \hfill\\
				\ce{C6H12O6} \textrightarrow \ \ce{2CO2} + 2\ce{CH3}-\ce{CH2}-\ce{OH} \hfill 2 ATP
		\end{description}

	\item Welche Enzyme werden für die Alkoholische Gärung benötigt und welches ist das Schlüsselenzym?

		Für die Alkoholische Gärung werden sowohl im Embden-Meyerhof-Parnas-Weg,
		als auch im Entner-Doudoroff-Weg folgende Enzyme benötigt:
		\begin{itemize}
			\item Pyruvat-Decaboxylase
			\item Ethanol-Dehydrogenas
		\end{itemize}
		
	\item Bei welchen Schritten werden bei der Alkoholischen Gärung ATP gebildet oder Reduktionsäquivalente verbraucht?
	\item Wie und warum unterscheidet sich die ATP-Ausbeute bei den alkoholischen Gärungen von Zymomonas mobilis und Saccharomyces cerevisiae?
	\item Welche Standorte besiedeln Milchsäurebakterien?
		
		Milchsäurebakterien finden sich in folgenden Habitaten:
		\begin{itemize}
			\item Pflanzen
			\item Milch, Milchprodukte
			\item Darm
			\item Schleimhäute
			\item Hautflora
		\end{itemize}

	\item Im Rahmen welcher Gärung taucht Methylmalonyl-CoA als Zwischenprodukt auf?
	\item Was ist das Schlüsselenzym der Milchsäuregärung?

		Kandidaten:
		\begin{itemize}
			\item Laktat-Dehydrogenase
			\item Phosphoketolasee
		\end{itemize}

	\item Benennen Sie Charakteristika von Bifidobakterien?
		
		Der Name der ergibts sich aus ihrer ``gespaltenen'', V- bzw. Y-Form.
		Ihr Habitat ist die Darmflora von Säuglingen,
		wohin sie durch die Muttermilch gelangen.
		In Kuhmilch sind sie jedoch nicht vorhanden.
		Es handelt sich um nicht aerotolerante Bakterien,
		die eine \ce{CO2} Atmosphäre benötigen.
		Bifidobakterien gären über den Phosphoketolase-Nebenweg.

	\item Wo verzweigen sich die Wege der homo- und der heterofermentativen Milchsäuregärung?
	\item Wie kommen die Löcher in den Emmentaler Käse?

		Die Löcher entstehen durch die \emph{Propionibakterien}.
		Sie vergären das Laktat auch zu \ce{CO2} (s.u.),
		welches die Löcher bedingt.
		\emph{Propionibakterien} sind gram-positive und	aerotolerant.
		Sie besiedeln den Pansen und Darm von Wiederkäuern und bauen Glukose über den
		Fruktose-1,6-bisphosphat-Weg ab, die weiter Gärung erfolgt mit:
		
		3 Laktat \textrightarrow \ 2 Propionat + 1 Acetat + \ce{CO2} + \ce{H2O}

	\item Nennen Sie Arten der Enterobacteriaceae.

		Die Enterobakterien sind gram-negative Stäbchenbakterien.
		Sie Sind peritrich begeißelt und somit beweglich.

		\begin{description}
			\item[E. coli] Darm
			\item[Enterobacter aerogenes] Boden
			\item[Erwinia] Pflanzen (pathogen)
		\end{description}

	\item Welches Verhältnis haben Enterobacteriaceae zum Sauerstoff?

		Seit dem sie sich von 2 Jahren getrennt haben, versuchen sie einen freundschaftliche Kontakt zu halten.

	\item Welche Gärprodukte werden von Enterobakterien gebildet?

		Folgenden Gärprodukte werden von Enterobakterien aus Hexosen gebildet:
		\begin{itemize}
			\item Ethanol
			\item 2,3-Butandiol
			\item Succinat
			\item Lactat
			\item Acetat
			\item Diacetyl
			\item Formiat
			\item \ce{H2}
			\item \ce{CO2}
		\end{itemize}

	\item Welche zwei Gärungstypen werden bei der gemischten Säuregärung unterschieden? Benennen sie hierfür repräsentative Vertreter!
	
		\emph{E. coli} verwendet einen Gärungstyp bei dem viel Säure und kein 2,3-Butandiol endsteht.
		Das Verhältnis von sauren Gärprodukten zu neutralen ist hier 4:1.
		Beim  zweiten Gärungstypen beträgt das Verhältnis 1:6.
		\emph{Enterobacter} und \emph{Klebsiella} beispielsweise,
		erzeigen hingegen wenig Säure und 2,3-Butandiol.

	\item Wie entstehen bei der gemischten Säuregärung Formiat, Wasserstoff und Kohlendioxid?
	\item Wie entsteht 2,3-Butandiol? und welche Enterobakterien produzieren es?
	\item Unter welchen Bedingungen wird die Bildung von 2,3-Butandiol begünstigt?
	\item Wie verteilen sich die Produkte bei der Vergärung von Glucose durch Escherichia coli bei Wachstum unter alkalischen bzw. sauren Bedingungen?
	\item Wie entsteht Succinat bei anaeroben Wachstum von Escherichia coli? Handelt es sich um einen Atmungs- oder Gärungsprozess?
\end{enumerate}
