\sectionmark{Biotechnologie I}

\section{Biotechnologie I}
\begin{enumerate}
	\item Nennen Sie Beispiele biotechnologisch hergestellter Produkte?
		
		 Biotechnologische Verfahrenn und ihre Produkte lassen sich in drei
		 zeitliche Phasen einteilen.
		 \begin{description}
		 	\item[Empirische Phase (>5000 Jahre)] \hfill \\
				Fermentation (Bier, Wein und Brot); mit natürlichen Stämmen
			\item[Einfache technische Verfahren (ab 1900)] \hfill \\
				wenige Metabolite (Säuren, Lösunsmittel, Riboflavin, Enzyme);
				einfache Screenings \textrightarrow einface Mutatenstämme
			\item[Systematische Verbesserung technischer Verfahren (ab 1945)] \hfill \\
				Antibiotika; Sauerstoffversorgung, Großtechnische Handhabung, Medienoptimierung
				Stammverbesserung über Mutagenese/Selektion und Rekombination/Selektion, Gentechnologie 
				%fas buzzwordbingo...
		\end{description}

	\item Was ist der Pasteur-Effekt?
			
		Der Pasteur-Effekt,
		bezeichnet den Effekt einer stark erhöhten Verstoffwechselung von D-Glucose in der Glykolyse,
		wenn kein Sauerstoff zur Verfügung steht.
		Dieses Phänomen tritt auch während der alkoholischen Gärung auf,
		die deshalb zum Teil in einer Sauerstoff-freien Umgebung durchgeführt werden muss.
			
	\item Wie ausgeprägt ist der Pasteur-Effekt bei \emph{Saccharomyces cerevisiae} und welche Bedeutung hat dies für biotechnologische Anwendung?
		
		Bei \emph{S. cerevisiae} ist der Pasteur-Effekt schächer ausgeprägt
		und so wird auch in einer Sauerstoffhaltigen Umgebung große Glucosemengen umgesetzt.
		Der unterschied zwischen aeroben und
		anaeroben Bedingungen beträgt nur 20 \begin{math}\mu\end{math}mol Glucose/min\textsuperscript{-1}g\textsuperscript{-1}
		(140 \begin{math}\mu\end{math}mol Glucose/min\textsuperscript{-1}g\textsuperscript{-1} unter aeroben Bedingungen).

		Durch diesen herabgestzten Pasteur-Effekt ist \emph{S. cerevisiae} gut für biotechnologische Prozesse zu verwenden.
		Andere Hefe-Arten sind deutlich empfindlicher gegenüber Sauerstoff
		und somit überwiegt der Vorteil der einfacheren Handhabbarkeit.
			
	\item Was ist der Unterschied zwischen ober- und untergärigen Hefen in Bezug auf die Bierproduktion?
	\label{item:oberuntergaerig}
		Bei untergärige Hefen sinken am Ende des Gärprozesses nach unten,
		obergärige Hefen steigen am Ende der Gärung an die Oberfläche.
		Untergäring Hefen werden zum Brauen von Pils und Lager verwendet,
		währen obergärige Hefen für Lager und Hefeweizen verwendet werden.

	\item Wie wird Bier hergestellt?

		Die Herstellung von Bier läuft in den Folgenenden Schritten ab:
		\begin{description}
			\item[Malzherstellung] \hfill \\
				Gerste wird zum Keimem gebracht um die Amylase-Enzyme zu erzeugen.
				Nach 5 Tagen wird das Keimen durch das sogenannte Darren abgebrochen.
				Dabei wird der enstandene Grünmalz auf 85\textdegree bis 100\textdegree erhitzt.
				Das Malz wird nun geschrotet (zerkleinert).
			\item[Maischen] \hfill \\
				Nun wird das Malz in einem Maischebottich mit 45\textdegree heißem Wasser vermengt.
				Die Stärke löst sich aus den Körnen und die Amylase wandelt die Stärke in vergärbare Maltose um.
				Unter ständigem rühren und Temperaturen knapp über 70\textdegree dauert der Maischprozess 3 bis 4 Stunden.
				Dieser Schritt ist entscheidend für den späteren Geschmack des Bieres.
			\item[Läutern] \hfill \\
				Nach dem der Brauer mit der Iodprobe ermittelt hat das sich keine Stärke mehr in der Maische befindet,
				wird die Bierwürze von den festen Bestandteilen getrennt.
				Im Läuterbottich sinken die festen Bestandteile nach unten und bilden den sog. Malzkuchen,
				durch den nun die Bierwürze in das Sudgefäß abläuft.
				Der Malzkuchen kann hinterher as Viehfutter verwendet werden.
			\item[Würzekochen] \hfill \\
				Der Bierwürze wird nun Hopfen zugegeben,
				was heute meist in Form von Exktrakt oder Pellets geschieht.
				Das Gemisch wird auf über 80\textdegree C aufgekocht,
				was zur Denaturierung der verbleibenden Amylase führt.
				Nun wird durch das Ausschlagen wiederum die endstandene Stammwürze abgeführt und
				anschließend abgekühlt.
			\item[Hefezugabe und Gärung] \hfill \\
				Nach dem die Stammwürze mit Eiswasser auf die für die Hefe notwendigen 5 bis 20\textdegree C abgekühlt wurde,
				wird selbige hinzugefügt (``Anstellen'').
				In 5 bis 8 Tagen vergärt etwa 70\% der Maltose zu Ethanol.
				Das Endstehenden \ce{CO2} wird abgesaugt und hinterher beim Abfüllen bzw. Zapfen verwendet.
				Je nach Bierart werden unter oder obergärige hefen verwendet (siehe Biotechnologie I, Frage \ref{item:oberuntergaerig}).
				Danach wird das Bier in Lagertanks gefüllt wo es drei bis fünf Monate nachgärt.
				Das dabei endstehende \ce{C02} wird nicht abgepumpt,
				wordurch es im Bier als Kohlensäure gebunden wird.
		\end{description}
			
	\item Welche Stoffwechselwege werden für die biotechnologische Produktion von Ethanol genutzt?
	\item Welche Organismen können 1,3-Propandiol produzieren?
	\item Wie wird Sauerkraut hergestellt?
	\item Wie wird Joghurt hergestellt?
	\item Welche Möglichkeiten gibt, um einen Stamm für einen biotechnologischen Prozess zu verbessern?
\end{enumerate}
