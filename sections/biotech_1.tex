\sectionmark{Biotechnologie I}

\section{Biotechnologie I}
\begin{enumerate}
	\item Nennen Sie Beispiele biotechnologisch hergestellter Produkte?
		
		 Biotechnologische Verfahrenn und ihre Produkte lassen sich in drei
		 zeitliche Phasen einteilen.
		 \begin{description}
		 	\item[Empirische Phase (>5000 Jahre)] \hfill \\
				Fermentation (Bier, Wein und Brot); mit natürlichen Stämmen
			\item[Einfache technische Verfahren (ab 1900)] \hfill \\
				wenige Metabolite (Säuren, Lösunsmittel, Riboflavin, Enzyme);
				einfache Screenings \texrightarrow einface Mutatenstämme
			\item[Systematische Verbesserung technischer Verfahren (ab 1945)] \hfill \\
				Antibiotika; Sauerstoffversorgung, Großtechnische Handhabung, Medienoptimierung
				Stammverbesserung über Mutagenese/Selektion und Rekombination/Selektion, Gentechnologie 
				%fas buzzwordbingo...
		\end{description}

	\item Was ist der Pasteur-Effekt?
	\item Wie ausgeprägt ist der Pasteur-Effekt bei Saccharomyces cerevisiae und welche Bedeutung hat dies für biotechnologische Anwendung?
	\item Was ist der Unterschied zwischen ober- und untergärigen Hefen in Bezug auf die Bierproduktion?
	\item Wie wird Bier hergestellt?
	\item Welche Stoffwechselwege werden für die biotechnologische Produktion von Ethanol genutzt?
	\item Welche Organismen können 1,3-Propandiol produzieren?
	\item Wie wird Sauerkraut hergestellt?
	\item Wie wird Joghurt hergestellt?
	\item Welche Möglichkeiten gibt, um einen Stamm für einen biotechnologischen Prozess zu verbessern?
\end{enumerate}
