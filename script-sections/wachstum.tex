\section{Wachstum}

\begin{description}
	\item[Anabolismus] \hfill \\
		Aufbauender, Energie verbrauchender Teil des Metabolismus.
	\item[Katabolismus] \hfill \\
		Abbauende, Energiefreisetzende Reaktion als Teil des Metabolismus.
\end{description}

\subsection{Elemente}

In den Tabellen \ref{tab:makroelemente} und \ref{tab:mikroelemente} befinden sich die
wichtigsten Makro- und Mikroelemente für das mikrobielle Wachstum.

Weiter Wichtige Sustanzen sind Vitamine,
welche von den Mikroorganismen aber auch selbst synthetisiert werden können.

\subsubsection*{Makroelemente}

\ce{CO2} ist bei Autotrophen der Kohlenstoff-Lieferant.

Stickstoff wird meist aus anorganoischen Quellen,
wie Ammoniak, Nitrat oder molekularem Stickstoff (\ce{N2}) gewonnen.

Phosphor ist entshceiden für die Synthese von Nucleinsäuren und Phospholipiden.

Schwefel wird als Strukturgeber für die Aminosäuren Cystein und Methionin, 
sowie Coenyzm A und Kiponsäuren.
Als Quelle dient Sulfat oder Sulfid.

Eisen spielt eine wichtige Rolle in der Atmung.
Es wird für die Cytochrome und Eisenschwefelproteinen beötigt,
wechle beim Elektronentransport mitwirken.

\begin{table}[h!]
	\begin{center}
		\begin{tabular}{l l} 
			\toprule
			Element	&	Bereitstellung	in der Natur \\
			\midrule
			C			&	\ce{CO2}, organische Stoffe \\
			H			&	\ce{H2O}, organische Stoffe \\
			O			&	\ce{H2O}, \ce{O2}, organische Stoffe \\
			N			&	\ce{NH3}, \ce{NO3-}, \ce{N2}, organische Stoffe \\
			P			&	Phosphat \\
			S			&	\ce{H2S}, organische Stoffe, Sulfide \\
			\midrule
			K			&	Kaliumsalze \\
			Mg			&	Magnesiumsalze \\
			Na			&	\ce{NaCl}, Natriumsalze \\
			Ca			&	Salze \\
			\midrule
			Fe			&	\ce{FeS}, Eisensalze \\
			\bottomrule
		\end{tabular}
		\caption{Makroelementen und ihre Quellen für Mikroorganismen}
		\label{tab:makroelemente}
	\end{center}
\end{table}

\subsubsection*{Mikroelemente}

Diese Substanzen weden nur in sehr geringen mengen benötigt,
ihr Vorhandensein ist jedoch entscheiden für das mikrobielle Wachstum.

\begin{table}[h!]
	\begin{center}
		\begin{tabular}{l l} 
			\toprule
			Element	&	Funktion in der Zelle \\
			\midrule
			Co			&	\ce{B12}, Transcaboxylase \\
			Cu			&	Atmung, CytC-Oxidase, Photosynthese\\
			Mn			&	Photosystem II, Superoxiddismutase \\
			Mo			&	Nitrogenase, Nitratreduktase, Formiat-DHG \\
			Ni			&	Hydorgenase, Co-F430 (Methanogene), \ce{CO}-Dehydrogenase \\
			Se			&	Formiat-DHG, Hydrogenase\\
			W			&	Formiat-DHG \\
			V			&	Vanadium-Nitrogenase \\
			Zn			&	Alkohol-DHG, RNA-/DNA-Polymerase \\
			Fe			&	Cytochrome, Katalasen, FeS-Proteine, alle Nitrogenasen \\
			\bottomrule
		\end{tabular}
		\caption{Mikroelementen und ihre Quellen für Mikroorganismen}
		\label{tab:mikroelemente}
	\end{center}
\end{table}

\subsection{Zellteilung}

Bei der Zweiteilung repliziert sich ein Mikroorganismus,
so dass nach dem Prozess zwei gleichwertige Organismen endstanden sind.

\begin{enumerate}
	\item DNA-Replikation
	\item Elongation des Zellkörpers
	\item Septumbildung
	\item Fertigstellung des Septums und Bildung getrennter Zellwände
	\item Trennung der Zellen
\end{enumerate}

Mit der Bildung des FtsZ-Moleküls wird der Prozess angestossen.
Es induziert die DNA-Replikation und Bildet das sogenannte Divisom,
den Zellteilungsapparat.
Der FtsZ-Ring läuft einmal an der Cytoplasmamembran um die Zelle herum und
markiert die Stelle der Trennung.

Der Ring trennt sich auf die beiden endstehenden Zellen auf und
in den Zwischenraum werden die neuen Zellaussenwände eingezogen,
bis das Cytosol der zellen abgeschlossen voneinander ist.

Durch bestimmte Proteine wird am Fts-Ring auch die neuen Zellwandstrukturen gebildet.
Dabei werden die Nahtstellen ``aufgeweicht'' und
verlängert um die Stabilität auch an diesen Stellen zu gewährleisten.

Dieser Prozess des ``Aufweichens'' wird durch Penicillin unterbunden,
was zu einem immer weiteren aufweichen und
damit zur Lyse der zelle führt.

\subsection{Wachstum von Populationen}

\begin{description}
	\item[Generation, Generationszeit] \hfill \\
		Zeit von einer Zweiteilung bis zur darauf Folgenden.

	\item[Exponentielles Wachstum] \hfill \\
		In der exponentiellen Phase vermehren sich bis die 
		Kapazitätsgrenzen erreicht ist und alle Nährstoffe des Habitats sind
		oder ein Stoffwechselprodukt im Medium das Wachstum begrenzt.

	\item[Wachstumsphasen] \hfill \\
		\begin{itemize}
			\item Lag	(+)
			\item Exponentiell	($2^x$)
			\item Satationär	(==)
			\item Absterben	(-)
		\end{itemize}

	\item[Messung] \hfill \\
		Beispielsweise durch Trübung.

\end{description}

\subsection{Umwelteinflüsse auf das Wachstum}

\subsubsection*{Temperatur}

Die Umgebungstemperatur hat einen großen Einfluss auf die Lebensvorgänge.
Mikroorgansimen sind meist auch an bestimmte Temperaturen angepasst
und können in anderen Umgebungen nicht wachsen.

Die Wachstumsrate nimmt,
ausgehend von der minimalen zum Leben notwendigen Temperatur,
linear zu,
bis sie das Optimum erreicht.
Danach fällt die Kurve stark ab und
erreicht bei einer nur geringen Temperatursteigerung das Maximum.
Die enzymatische Aktivität läuft im bereich des Optimusm am besten ab,
da höhere Temperatur die Reaktionsgeschwindigkeit erhöhen.
Wir das Optimum überschritten,
beginnen Enzyme an zu denaturieren und eine thermisch bedingte Lyse setzt ein.
Unterhalb des Temperatur-Minimus ist die Membran zu starr 
und Transportprozess können nicht ablauf.
Zusätzlich kann die Membran gelförmig werden,
und instabil werden.

In Tabelle \ref{tab:tempGruppen} findet sich eine Überisch über die verschiedenen Temperaturen
und die Bezeichnung für die in diesem Bereich lebendend Mikroorganismen.

\begin{table}
	\centering
	\begin{tabular}{l l l}
		\toprule
		Bezeichnung		& optimale Wachstumstemperatur	& Vertreter \\
		\midrule
		Psychrophil		&	4°		&\emph{Polaromonas vaculota}\\
		Mesophil			&	39°	&\emph{Escherichia coli}\\    
		Thermophil		&	60°	&\emph{Bacillus stearothermophilus}\\
		Hyperthermophil&	88°	&\emph{Thermococcus celer}\\  
		Hyperthermophil&	106°	&\emph{Pyrolobus fumarii}\\   
		\bottomrule
	\end{tabular}
	\caption{Übersicht über Temperaturoptima typischer Mikroorganismen.}
	\label{tab:tempGruppen}
\end{table}

Bei Extremophilen ist in jedemfall eine Anpassung der Membranen nötig.
Entweder eine Vertärkung der Festigkeit (Hyperthermophile)
oder eine Verbesserng der Fluidität (Psychrophile).

\subsubsection*{Weiter Umwelteinflüsse}

\begin{description}
	\item[pH-Wert]			\hfill	\\
		Die meisten Mikroorganismen wachsen in einem pH-Bereich von 2 bis 3 Einheiten.
		Wobei die meisten Habitate einen pH-Wert zwischen 5 und 9 haben
		und das Optimum der meisten Organismen liegt in diesem Bereich.

		Wenige Organismen haben ihr Optimum bei einem pH-Wert von weniger als 2.
		Sie werden als Acidophile bezeichnet.
		Als Alkaliphile werden hingegen alle Organismen bezeichnet,
		die ihr Optimum bei einem pH-Wert von mehr als 9 haben.

		Der intrazelluäre pH-Wert ist dabei dem der Umgebeung sehr ähnlich.
	
	\item[Osmolarität]	\hfill	\\
		\begin{itemize}
			\item Nicht Halophil
			\item Halotolerant
			\item Halophil	(schwach:1-6\%; gemäßigt:7-15\%)
			\item Extrem Halophil 
		\end{itemize}

		Die sind die Kategorien in die Sich Mikroorganismen einsortieren lassen,
		im Bezug auf die Natriumcloridkonzentration.
		Extrem Halophile tolerieren \ce{NaCl}-Koneztration von bis zu 30\%.
		Marine Organismen haben ihr Optimum bei etwa 3\%.
	
	\item[Sauerstoff]		\hfill	\\
		Hier erfolgt die Teilung in Aerobier und Anaerobier.
		Wobei es sowohl fakultative,
		als auch obligate.

		Die Anapassung an Sauerstoff sind gravierend,
		da sie auch den Metabolismus beeinflussen.
		So fermentieren Anaerobe,
		wohingegen Aerobier Nährstoffe veratmen.

		Die toxischen Sauerstoffvarianten wie Wasserstoffperoxid \ce{H2O2},
		Superoxid \ce{O2-} oder das Hydroxydradikal \ce{OH-}
		müssen von Mirkoorgansimen abgebaut werden können.
		Diese Substanzen sind sehr reaktiv und können quasi alle Moleküle des Cytosols oxidieren.
		Daher gibt es spezielle Enzyme die ihren Abbau katakylsieren.
		\begin{table}
			\centering
			\begin{tabular}{l l}
				\toprule
				Enzym						&		Reaktion		\\
				\midrule
				Katalase					&	\textcolor{red}{\ce{H2O2}} + \textcolor{red}{\ce{H2O2}} 
									\textrightarrow 2\ \ce{H2O} + \ce{O2}							\\
				Peroxidase				&	\textcolor{red}{\ce{H2O2}} + \ce{NADH} + \ce{H+}
									\textrightarrow 2\ \ce{H2O} + \ce{NAD+}							\\
				Superoxiddismutase	&	\textcolor{red}{\ce{O2-}} + \textcolor{red}{\ce{O2-}} 
									\textrightarrow 2\ \ce{H+} + \textcolor{red}{\ce{H2O2}} +  \ce{O2}	\\
									Superoxidreduktase	&	\textcolor{red}{\ce{O2-}} + 2 \ce{H+} + cyt\slshape{c}\textsubscript{reduziert}
									\textrightarrow \textcolor{red}{\ce{H2O2}} + + cyt\slshape{c}\textsubscript{oxidiert} \\
				\bottomrule
			\end{tabular}
			\caption{\textcolor{red}{Toxische Sauerstoffspzeies} und ihre verarbeitenden Enzyme.}
			\label{tab:toxO2}
		\end{table}
\end{description}
