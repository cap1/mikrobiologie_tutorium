\section{Wachstum}

\begin{description}
	\item[Anabolismus] \hfill \\
		Aufbauender, Energie verbrauchender Teil des Metabolismus.
	\item[Katabolismus] \hfill \\
		Abbauende, Energiefreisetzende Reaktion als Teil des Metabolismus.
\end{description}

\subsection{Elemente}

In den Tabellen \ref{tab:makroelemente} und \ref{tab:mikroelemente} befinden sich die
wichtigsten Makro- und Mikroelemente für das mikrobielle Wachstum.

Weiter Wichtige Sustanzen sind Vitamine,
welche von den Mikroorganismen aber auch selbst synthetisiert werden können.

\subsubsection*{Makroelemente}

\ce{CO2} ist bei Autotrophen der Kohlenstoff-Lieferant.

Stickstoff wird meist aus anorganoischen Quellen,
wie Ammoniak, Nitrat oder molekularem Stickstoff (\ce{N2}) gewonnen.

Phosphor ist entshceiden für die Synthese von Nucleinsäuren und Phospholipiden.

Schwefel wird als Strukturgeber für die Aminosäuren Cystein und Methionin, 
sowie Coenyzm A und Kiponsäuren.
Als Quelle dient Sulfat oder Sulfid.

Eisen spielt eine wichtige Rolle in der Atmung.
Es wird für die Cytochrome und Eisenschwefelproteinen beötigt,
wechle beim Elektronentransport mitwirken.

\begin{table}[h!]
	\begin{center}
		\begin{tabular}{l l} 
			\toprule
			Element	&	Bereitstellung	in der Natur \\
			\midrule
			C			&	\ce{CO2}, organische Stoffe \\
			H			&	\ce{H2O}, organische Stoffe \\
			O			&	\ce{H2O}, \ce{O2}, organische Stoffe \\
			N			&	\ce{NH3}, \ce{NO3-}, \ce{N2}, organische Stoffe \\
			P			&	Phosphat \\
			S			&	\ce{H2S}, organische Stoffe, Sulfide \\
			\midrule
			K			&	Kaliumsalze \\
			Mg			&	Magnesiumsalze \\
			Na			&	\ce{NaCl}, Natriumsalze \\
			Ca			&	Salze \\
			\midrule
			Fe			&	\ce{FeS}, Eisensalze \\
			\bottomrule
		\end{tabular}
		\caption{Makroelementen und ihre Quellen für Mikroorganismen}
		\label{tab:makroelemente}
	\end{center}
\end{table}

\subsubsection*{Mikroelemente}

Diese Substanzen weden nur in sehr geringen mengen benötigt,
ihr Vorhandensein ist jedoch entscheiden für das mikrobielle Wachstum.

\begin{table}[h!]
	\begin{center}
		\begin{tabular}{l l} 
			\toprule
			Element	&	Funktion in der Zelle \\
			\midrule
			Co			&	\ce{B12}, Transcaboxylase \\
			Cu			&	Atmung, CytC-Oxidase, Photosynthese\\
			Mn			&	Photosystem II, Superoxiddismutase \\
			Mo			&	Nitrogenase, Nitratreduktase, Formiat-DHG \\
			Ni			&	Hydorgenase, Co-F430 (Methanogene), \ce{CO}-Dehydrogenase \\
			Se			&	Formiat-DHG, Hydrogenase\\
			W			&	Formiat-DHG \\
			V			&	Vanadium-Nitrogenase \\
			Zn			&	Alkohol-DHG, RNA-/DNA-Polymerase \\
			Fe			&	Cytochrome, Katalasen, FeS-Proteine, alle Nitrogenasen \\
			\bottomrule
		\end{tabular}
		\caption{Mikroelementen und ihre Quellen für Mikroorganismen}
		\label{tab:mikroelemente}
	\end{center}
\end{table}

\subsection{Zellteilung}

Bei der Zweiteilung repliziert sich ein Mikroorganismus,
so dass nach dem Prozess zwei gleichwertige Organismen endstanden sind.

\begin{enumerate}
	\item DNA-Replikation
	\item Elongation des Zellkörpers
	\item Septumbildung
	\item Fertigstellung des Septums und Bildung getrennter Zellwände
	\item Trennung der Zellen
\end{enumerate}

Mit der Bildung des FtsZ-Moleküls wird der Prozess angestossen.
Es induziert die DNA-Replikation und Bildet das sogenannte Divisom,
den Zellteilungsapparat.
Der FtsZ-Ring läuft einmal an der Cytoplasmamembran um die Zelle herum und
markiert die Stelle der Trennung.

Der Ring trennt sich auf die beiden endstehenden Zellen auf und
in den Zwischenraum werden die neuen Zellaussenwände eingezogen,
bis das Cytosol der zellen abgeschlossen voneinander ist.

Durch bestimmte Proteine wird am Fts-Ring auch die neuen Zellwandstrukturen gebildet.
Dabei werden die Nahtstellen ``aufgeweicht'' und
verlängert um die Stabilität auch an diesen Stellen zu gewährleisten.

Dieser Prozess des ``Aufweichens'' wird durch Penicillin unterbunden,
was zu einem immer weiteren aufweichen und
damit zur Lyse der zelle führt.

\subsection{Wachstum von Populationen}

\begin{description}
	\item[Generation, Generationszeit] \hfill \\
		Zeit von einer Zweiteilung bis zur darauf Folgenden.

	\item[Exponentielles Wachstum] \hfill \\
		In der exponentiellen Phase vermehren sich bis die 
		Kapazitätsgrenzen erreicht ist und alle Nährstoffe des Habitats sind
		oder ein Stoffwechselprodukt im Medium das Wachstum begrenzt.

	\item[Wachstumsphasen] \hfill \\
		\begin{itemize}
			\item Lag	(+)
			\item Exponentiell	($2^x$)
			\item Satationär	(==)
			\item Absterben	(-)
		\end{itemize}

	\item[Messung] \hfill \\
		Beispielsweise durch Trübung.

\end{description}

\subsection{Umwelteinflüsse auf das Wachstum}
