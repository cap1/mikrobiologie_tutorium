\section{Biotechnologie}

\begin{description}
	\item[Pasteur-Effekt] \hfill \\
		Mikroorganismen Verstoffwechseln D-Glucose wesentlich stärker in der Gykolyse,
		wenn keine Sauerstoff zur Verfügung steht.
		Der Effekt tritt auch bei der alkoholischen Gärung auf,
		welche aus diesem Grund auch in einer Sauerstofffreien Umgebung durchgeführt wird.

		Bei \emph{S. cerevivsiae} ist der Pasteur-Effekt sehr schwach ausgeprägt,
		was die Hefe für biotechnologische Verfahren interessant macht.

	\item[Alkohol-Produktion] \hfill \\
		Bei der industriellen Herstellung von Ethanol kommt eine ``normale'' gemischte Säuregärung zum Einsatz,
		welche jedoch in Richtung Ethanolproduktion verschoben wird.
		Durch die Eiführung eines ``pet-Operons'' wird die Produktion von Succinat unterbunden.
		Verwendet werden: \emph{S. cerevisiae} und \emph{Zymomonas mobilis}.

	\item[1,3-Propandiol Produktion] \hfill \\
		\begin{itemize}
			\item[Enterobakterien] 
				\emph{Klebsiella pneumononiae},
				\emph{Klebsiella oxytoca},
				\emph{Citrobacter freundii},
				\emph{Citrobacter intermedium}, 
				nicht jedoch \emph{E. coli}
			\item[Clostriedien] 
				\emph{Clostridium pasteurianum},
				\emph{Clostridium butyricum},
				\emph{Clostridium perfingens}
			\item[Andere]
				\emph{Acetobacterium woodii},
				\emph{Acetobacterium carbinolicum},
				\emph{Ilyobacter polytropus},
				\emph{Lactobacillus reuteri},
				\emph{Lactobacillus bervis},
				\emph{Lactobacillus buchneri}
		\end{itemize}

	\item[Prozessoptimierung] \hfill \\
		\begin{itemize}
			\item Screening un Selektion von Stämmen
			\item Gentechnische Veränderung
			\item Optimierung der Bedingungen (Substratzugabe, Produktendtzug, \ldots)
		\end{itemize}

	\item[unvollständige Oxidation] \hfill \\
		Meist mit der Wachstumsphase assoziierte Form der \ce{O2}-Oxidation,
		bei der nicht \ce{CO2} das Ziel ist, sondern ein Zwischenprodukt.

		Durch die unvollständige Oxidation werden biotenologisch hergestellt:
		
		\begin{itemize}
			\item Acetat
			\item Gluconat
			\item Fumarat
			\item Citrat und andere organische Säuren
			\item Aminosäuren
			\item Alkohole
		\end{itemize}

	\item[Biokonversion] \hfill \\
		C-Metabolismus während der stationären Phase. %TODO mehr!

	\item[Sekundär Stoffwechsel] \hfill \\
		Stoffwechsel-Weg zur Erzeugung bestimmter Stoffe die seltener benötigt werden
		(z.B. Antibiotika).

	\item[Tropophase] \hfill \\
		Die Wachstumsphase einer Zelle in der hauptsächlich der Primärmetabolit erzeugt wird.

	\item[Idiophase] \hfill \\
		Die Produktionssphase einer Zelle in der hauptsächlich der Sekundärmetabolit erzeugt wird.

	\item[Angrifforte für Antibiotika] \hfill \\

		\begin{table}[ht!]
		\begin{center}
			\begin{tabular}{p{4cm} p{3cm}} 
		\toprule
			Wirkort	&	Antibiotika \\
			\midrule
			DNS-Replikation		&	Nitroimidazole \\
										&	Fluorcinolone 	\\
			Zellwandbiosynthese	&	\begin{math}\beta\end{math}-Lactame \\
			\multirow{3}{*}{}		&	Glycopeptide 	\\
										&	Bacitaricin 	\\
										&	Cycloserin 		\\
			Folsäurestoffwechsel	&	Trimethoprim	\\
										&	Sulfonamide 	\\
			Proteinbiosynthese	&	Tetracycline \\
			\multirow{5}{*}{}		&	Makrolide 	\\
										&	Aminoglycoside 	\\
										&	Lincosamide 	\\
										&	Oxazolidinone 	\\
										&	Streptogramine 	\\
										&	Chloramphenicol 	\\
			RNS-Polymerase			&	Rifamycine \\
		\bottomrule
		\end{tabular}
		\caption{Angriffsvektoren und die passenden Antiobiotika.}
		\label{tab:wirkorteantibiose}
		\end{center}
		\end{table}

\end{description}
