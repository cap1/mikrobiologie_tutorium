\section{Systematik}

\subsection{Essigsäurebakterien}

\subsubsection*{Gruppen}
\begin{description}
	\item[Glucanobacter \& Acetobacter] \hfill \\
	\begin{itemize}
		\item Gram-Negativ
		\item bewegliche Stäbchen
		\item obligat Arobier
		\item unvollständige Oxidation von Alkoholen und Glukose
	\end{itemize}
\end{description}

\subsubsection*{Fähigkeiten}
\begin{description}
	\item[Ethanol \textrightarrow Acetat Umsetzung]\hfill\\
		\begin{enumerate}
			\item Ethanolhydorogenase(Ethanol) \textrightarrow \ Acetaldehyd
			\item Aldehyddehyrogenase(Acetaldehyd) \textrightarrow \ Essigsäure
		\end{enumerate}
	\item[Ascorbinsäuresynthese] \hfill \\
		\emph{Acetobacter suboxydans} wandelt in der Reichstei-Güssner-Synthese
		D-Sorbit in L-Sorbose um.
		Dabei endsteht \ce{NADH2}.
		
\end{description}

\subsection{Milchsäurebakterien}

\subsubsection*{Typische Vertreter}
\begin{itemize}
	\item Lactobacillus
	\item Leuconostoc
	\item Streptococcus
	\item Bifidobacterium
\end{itemize}

\subsubsection*{Eigenschaften}

\emph{Bifidobakterien}
\begin{itemize}
	\item bifidus: gespalten \textrightarrow V-Form, Y-Form
	\item in der Darmflora von Säuglingen, braucht N-Glukosamin (nicht in Kuhmilch)
	\item \textsl{nicht aerotolerant}
	\item braucht \ce{CO2}-haltige Atmosphäre
	\item starke Phosphoketolaeaktivität
	\item Phosphoketolase-Nebenweg
\end{itemize}

\emph{Propionibaterien}
\begin{itemize}
	\item Im Pansen \& Darm von Wiederkäuern
	\item Emmentalerkäse \textrightarrow ``Löcher'' aus \ce{CO2}
	\item Glukoseabbau FBP-Weg
	\item 3 Laktat \textrightarrow 2 Propionat + Acetat + \ce{CO2} + \ce{H2O}
\end{itemize}

\emph{Allgemein}
\begin{itemize}
	\item Scheiden große Mengen Lactat aus
	\item Säuretolerant bis pH 3,5
	\item diverse Morphologie
	\item Gram-positiv, keine Sporen
	\item Meist unbeweglich
	\item Obligate Gärer, keine Hämbiosynthese
	\item Meist aerotolerant
	\item können Milchzucker spalten\\ $\beta$-Galaktosidase(Lactose) \textrightarrow Gulcose + Galactose
\end{itemize}

\subsubsection*{Habitate}
\begin{itemize}
	\item Pflanzen
	\item Milch (-Produkte)
	\item Darm / Schleimhäute
	\item Hautflora, verhinder Besiedlung von Pathogenen
\end{itemize}

\subsection{Enterobakterien}

\subsection{Clostridien}
%Gaerung2 S. 5
